%%%%%%%%%%%%%%%%%%%%%%%%%%%%%%%%%%%%%%%%%
% Masters/Doctoral Thesis 
% LaTeX Template
% Version 2.4 (22/11/16)
%
% This template has been downloaded from:
% http://www.LaTeXTemplates.com
%
% Version 2.x major modifications by:
% Vel (vel@latextemplates.com)
%
% This template is based on a template by:
% Steve Gunn (http://users.ecs.soton.ac.uk/srg/softwaretools/document/templates/)
% Sunil Patel (http://www.sunilpatel.co.uk/thesis-template/)
%
% Template license:
% CC BY-NC-SA 3.0 (http://creativecommons.org/licenses/by-nc-sa/3.0/)
%
%%%%%%%%%%%%%%%%%%%%%%%%%%%%%%%%%%%%%%%%%

%----------------------------------------------------------------------------------------
%	PACKAGES AND OTHER DOCUMENT CONFIGURATIONS
%----------------------------------------------------------------------------------------

\documentclass[
11pt, % The default document font size, options: 10pt, 11pt, 12pt
%oneside, % Two side (alternating margins) for binding by default, uncomment to switch to one side
english, % ngerman for German
singlespacing, % Single line spacing, alternatives: onehalfspacing or doublespacing
%draft, % Uncomment to enable draft mode (no pictures, no links, overfull hboxes indicated)
%nolistspacing, % If the document is onehalfspacing or doublespacing, uncomment this to set spacing in lists to single
%liststotoc, % Uncomment to add the list of figures/tables/etc to the table of contents
%toctotoc, % Uncomment to add the main table of contents to the table of contents
%parskip, % Uncomment to add space between paragraphs
%nohyperref, % Uncomment to not load the hyperref package
headsepline, % Uncomment to get a line under the header
%chapterinoneline, % Uncomment to place the chapter title next to the number on one line
%consistentlayout, % Uncomment to change the layout of the declaration, abstract and acknowledgements pages to match the default layout
]{mediaproject} % The class file specifying the document structure

\usepackage[utf8]{inputenc} % Required for inputting international characters
\usepackage[T1]{fontenc} % Output font encoding for international characters

%----------------------------------------------
% custom packages
\setlength{\parindent}{8ex}
\usepackage{subcaption}
\usepackage{pgfgantt}
\usepackage{xcolor}
\usepackage[utf8]{inputenc}

\definecolor{barblue}{RGB}{153,204,254}
\definecolor{groupblue}{RGB}{51,102,254}
\definecolor{linkred}{RGB}{165,0,33} 





\newlength{\tempdima}
\newcommand{\rowname}[1]% #1 = text
{\rotatebox{90}{\makebox[\tempdima][c]{\textbf{#1}}}}

%\newcounter{subfigure}[figure]
%\renewcommand{\thesubfigure}{\alph{subfigure}}
%\newcommand{\mycaption}[1]% #1 = caption
%{\refstepcounter{subfigure}\textbf{(\thesubfigure) }{\ignorespaces #1}}





%----------------------------------------------

\usepackage{svg}
\usepackage{graphicx}
\graphicspath{ {./Figures/} }

\usepackage{palatino} % Use the Palatino font by default
\usepackage[backend=bibtex,natbib=true]{biblatex} % Use the bibtex backend with the authoryear citation style (which resembles APA)

\addbibresource{reference.bib} % The filename of the bibliography

\usepackage[autostyle=true]{csquotes} % Required to generate language-dependent quotes in the bibliography
\newcommand{\shivam}[1]{\textcolor{blue}{#1}}
\newcommand{\nadacn}[1]{{\color{orange} #1}}
\newcommand{\notice}[1]{{\color{red} #1}}
\newcommand{\note}[1]{{\color{green} #1}}

%----------------------------------------------------------------------------------------
%	MARGIN SETTINGS
%----------------------------------------------------------------------------------------

\geometry{
	paper=a4paper, % Change to letterpaper for US letter
	inner=2.5cm, % Inner margin
	outer=3.8cm, % Outer margin
	bindingoffset=.5cm, % Binding offset
	top=1.5cm, % Top margin
	bottom=1.5cm, % Bottom margin
	%showframe, % Uncomment to show how the type block is set on the page
}



%----------------------------------------------------------------------------------------
%	THESIS INFORMATION
%----------------------------------------------------------------------------------------
\thesistitle{Monocular Depth Estimation Of Indoor Environment Using Neural Network} % Your thesis title, this is used in the title and abstract, print it elsewhere with \ttitle
\supervisor{Prof. Dr. Wolfgang Brol} % Your supervisor's name, this is used in the title page, print it elsewhere with \supname
\examiner{Christian Kunert, MSc} % Your examiner's name, this is not currently used anywhere in the template, print it elsewhere with \examname
\degree{Doctor of Philosophy} % Your degree name, this is used in the title page and abstract, print it elsewhere with \degreename
\author{Christon-Ragavan Nadar and Shivam Sani} % Your name, this is used in the title page and abstract, print it elsewhere with \authorname
\addresses{} % Your address, this is not currently used anywhere in the template, print it elsewhere with \addressname
\keywords{} % Keywords for your thesis, this is not currently used anywhere in the template, print it elsewhere with \keywordnames
\university{\href{http://www.tu-ilmenau.de}{Ilmenau University of Technology}} % Your university's name and URL, this is used in the title page and abstract, print it elsewhere with \univname
\department{\href{http://www.tu-ilmenau.de/wm}{Faculty of Economic Sciences and Media}}
\group{\href{http://www.tu-ilmenau.de/vwdg}{Virtual Worlds and Digital Games Group}} % Your research group's name and URL, this is used in the title page, print it elsewhere with \groupname

\AtBeginDocument{
\hypersetup{pdftitle=\ttitle} % Set the PDF's title to your title
\hypersetup{pdfauthor=\authorname} % Set the PDF's author to your name
\hypersetup{pdfkeywords=\keywordnames} % Set the PDF's keywords to your keywords
}

\begin{document}

\frontmatter % Use roman page numbering style (i, ii, iii, iv...) for the pre-content pages

\pagestyle{plain} % Default to the plain heading style until the thesis style is called for the body content

%----------------------------------------------------------------------------------------
%	TITLE PAGE
%----------------------------------------------------------------------------------------

\begin{titlepage}
\begin{center}

\vspace*{.06\textheight}
{\scshape\LARGE \univname\par}\vspace{1.5cm} % University name
\textsc{\Large Media Project}\\[0.5cm] % Thesis type

\HRule \\[0.4cm] % Horizontal line
{\huge \bfseries \ttitle\par}\vspace{0.4cm}  % Thesis title
\HRule \\[1.5cm] % Horizontal line
 


\begin{minipage}[t]{\textwidth}
        \centering
        \emph{Authors:}\\
        \href{http://www.christonragavan.com}{\authorname} % Author name - remove the \href bracket to remove the link

\end{minipage}


\vspace*{.06\textheight}


\large \textit{Final report for a media project}\\[0.3cm] % University requirement text
\textit{in the}\\[0.4cm]
\groupname\\\deptname\\[2cm] % Research group name and department name
 
\vfill

{\large \today}\\[4cm] % Date
%\includegraphics{Logo} % University/department logo - uncomment to place it
 
 \begin{minipage}[t]{0.4\textwidth}
     \begin{flushleft} \large
         \emph{Supervisor:}\\
         \href{https://www.tu-ilmenau.de/vwds/team/wolfgang-broll/}{\supname}
     \end{flushleft}
 \end{minipage}
 \begin{minipage}[t]{0.4\textwidth}
     \begin{flushright} \large
         \emph{Advisor:} \\
         \href{https://www.tu-ilmenau.de/vwds/team/christian-kunert/}{\examname} 
     \end{flushright}
 \end{minipage}\\[3cm]

\vfill
\end{center}
\end{titlepage}

%----------------------------------------------------------------------------------------
%	DECLARATION PAGE
%----------------------------------------------------------------------------------------

\begin{declaration}
\addchaptertocentry{\authorshipname} % Add the declaration to the table of contents
\noindent WE/I, \authorname, declare that this report titled, \enquote{\ttitle} and the work presented in it are our/my own. We/I confirm that:

\begin{itemize} 
\item This work was done wholly or mainly while in candidature for a research degree at this University.
\item Where any part of this work has previously been submitted at this University or any other institution, this has been clearly stated.
\item Where We/I have consulted the published work of others, this is always clearly attributed.
\item Where We/I have quoted from the work of others, the source is always given. With the exception of such quotations, this report describes entirely our/my work.
\item I have acknowledged all main sources of help.
\item Where the thesis is based on work done by ourself/myself jointly with others, We/I have made clear exactly what was done by others and what We/I have contributed myself.
\end{itemize}
 
\noindent Signed:\\
\rule[0.5em]{25em}{0.5pt} % This prints a line for the signature
 
\noindent Date:\\
\rule[0.5em]{25em}{0.5pt} % This prints a line to write the date
\end{declaration}

\cleardoublepage

%----------------------------------------------------------------------------------------
%	QUOTATION PAGE
%----------------------------------------------------------------------------------------

\vspace*{0.2\textheight}

\begin{center}
\noindent\enquote{\itshape In God we trust, all others bring data.}\bigbreak  
\end{center}
\hfill William Edwards Deming (1900-1993).

%----------------------------------------------------------------------------------------
%	ABSTRACT PAGE
%----------------------------------------------------------------------------------------
\begin{abstract}
	\addchaptertocentry{\abstractname} % Add the abstract to the table of contents
	Depth estimation is one of the basic building block for scene understanding. Especially in the case of monocular depth estimation using Neural Network many approaches has been done. These approaches can be highly hardware dependent which results into an task and environment specific optimizing problem. Finding a generalized model with the consideration of different hardware properties of sensors and platforms could be challenging. In this study we try address this problem, with a primary objective of achieving a Neural Network which can replace a depth sensor, in our case it is Structure Sensor by Occiptal integrated with IPad device. In this study, first we propose a novel Neural Network model architecture which is much smaller with respect to trainable parameter when compared to the state of the art models. Due to lack of data and trainable parameter the proposed system did not out perform the existing methods. Secondly we propose another novel idea for a input feature representation for Neural Network, intuitively speaking, to train a Neural Network which can mimic the structural sensor. Our idea was to make the model learn the holes (dead pixels) generated by the sensor and there by we can a Neural Network can completely replace the existing hardware model. Which experimenting We found that, this approach results into a artifacts of generating intermediate pixels between the actual holes and next neighboring non-hole pixel value. This is a challenge when considering a end to end approach, where careful post processing methods could be needed to eradicate such artifacts. Thirdly, we generated a new dataset form the Structure Sensor to train our Neural Network. Lastly, we did case study on our proposed model with respect to the existing systems trying to understand various factors influencing the performance of the network. Keeping the primary goal in focus we tried different experimental configuration setup to find the best working model suitable for the given tasks by various transfer learning methods. Our experiments have proven to find the best working model showing outstanding performance with the mean accuracy of \textbf{98\%}. Not only we achieved the best model for this task, we also tried to analyze various factors involved which influenced the performance of the first and second ideas mentioned before. 
	
\end{abstract}


%----------------------------------------------------------------------------------------
%	ACKNOWLEDGEMENTS
%----------------------------------------------------------------------------------------

\begin{acknowledgements}
\addchaptertocentry{\acknowledgementname} % Add the acknowledgements to the table of contents
We would not like to thank Professor Broll and the department of Virtual Worlds and Digital Games Group who have given an opportunity to do this study.\\

We would like to show our initial gratitude towards our project advisor Christian Kunert, M. Sc. who created an supportive and encouraging atmosphere during our project. We are very pleased to for his availability and brainstorming over various ideas and methods. Also we are thankfull for Tobias Schwandt, M. Sc. for his ever present support during this entire study process. My friend Andrew Ng, M. Sc. who have always been my great encouragement and support. We would like to  

Not to forget, a hugh thank you note towards various YouTube content creators and blog writters who have played an important role in understanding various Network Concepts, without which this journey would have not been easier. 


\end{acknowledgements}

%----------------------------------------------------------------------------------------
%	LIST OF CONTENTS/FIGURES/TABLES PAGES
%----------------------------------------------------------------------------------------

\tableofcontents % Prints the main table of contents

\listoffigures % Prints the list of figures

\listoftables % Prints the list of tables

%----------------------------------------------------------------------------------------
%	ABBREVIATIONS
%----------------------------------------------------------------------------------------

%\begin{abbreviations}{ll} % Include a list of abbreviations (a table of two columns)
%
%\textbf{LAH} & \textbf{L}ist \textbf{A}bbreviations \textbf{H}ere\\
%\textbf{WSF} & \textbf{W}hat (it) \textbf{S}tands \textbf{F}or\\
%
%\end{abbreviations}

%----------------------------------------------------------------------------------------
%	PHYSICAL CONSTANTS/OTHER DEFINITIONS
%----------------------------------------------------------------------------------------

%\begin{constants}{lr@{${}={}$}l} % The list of physical constants is a three column table
%
%% The \SI{}{} command is provided by the siunitx package, see its documentation for instructions on how to use it
%
%Speed of Light & $c_{0}$ & \SI{2.99792458e8}{\meter\per\second} (exact)\\
%Constant Name & $Symbol$ & $Constant Value$ with units\\
%
%\end{constants}

%----------------------------------------------------------------------------------------
%	SYMBOLS
%----------------------------------------------------------------------------------------

%\begin{symbols}{lll} % Include a list of Symbols (a three column table)
%
%$a$ & distance & \si{\meter} \\
%$P$ & power & \si{\watt} (\si{\joule\per\second}) \\
%%Symbol & Name & Unit \\
%
%\addlinespace % Gap to separate the Roman symbols from the Greek
%
%$\omega$ & angular frequency & \si{\radian} \\
%
%\end{symbols}

%----------------------------------------------------------------------------------------
%	DEDICATION
%----------------------------------------------------------------------------------------

%\dedicatory{For/Dedicated to/To my\ldots} 

%----------------------------------------------------------------------------------------
%	THESIS CONTENT - CHAPTERS
%----------------------------------------------------------------------------------------

\mainmatter % Begin numeric (1,2,3...) page numbering

\pagestyle{thesis} % Return the page headers back to the "thesis" style

% Include the chapters of the thesis as separate files from the Chapters folder
% Uncomment the lines as you write the chapters

% Chapter 1

\chapter{Introduction} % Main chapter title

\label{Chapter1} % For referencing the chapter elsewhere, use \ref{Chapter1} 

%----------------------------------------------------------------------------------------
%1 page: Why you do this topic? Relevancy. What's the problem?
%1-2 page(s): What are you doing?
%1 page: Hence, this thesis tries to answer the following research question(s):
%total 4
%----------------------------------------------------------------------------------------

% Define some commands to keep the formatting separated from the content 
\newcommand{\keyword}[1]{\textbf{#1}}
\newcommand{\tabhead}[1]{\textbf{#1}}
\newcommand{\code}[1]{\texttt{#1}}
\newcommand{\file}[1]{\texttt{\bfseries#1}}
\newcommand{\option}[1]{\texttt{\itshape#1}}

%----------------------------------------------------------------------------------------


%1 page: Why you do this topic? Relevancy. What's the problem?
%----------------------------------------------------------------------------------------

Scene understanding is long existing and an interesting area for computer vision. This is highly depended on human visual three dimensional (3D) perception. Understanding the structural dependencies is one of the important task for 3D scene understanding and for its reconstruction, which is basically recovering the range and orientation of the surface and object to mimic the humans visual behaviour \cite{barnard1982computational}. This creates a multiple opportunities for various multimedia computing applications. Most of the time the oblique projection of 3D data into two dimensional(2D) plan is enough for many of our application like 3D displays, movies, games or structural building representation. But in contrast applications for immersive augmented and virtual reality entertainment, tracking people or objects and observing their activities for robotics or automation for various inductrial and autonomous driving systems etc., having a third dimension data which is depth could be vital. This shows the importance of having 3D depth information for these area of applications. \\

As a solution there are many techniques used to derive depth information which can be categorized into 3. First, dual camera method \cite{li2009dual} second,  dual pixel method \cite{martinello2015dual, choi2017all} and  third sensors based methods \cite{salvi2004pattern}. In extension various methods of estimating depth from focus \cite{grossmann1987depth}, stereo vision \cite{bulthoff1988integration}, and depth from motion \cite{ullman1979interpretation} where also applied. We will discuss more in detail in section \ref{Chapter3:RelatedWork_EarlyApproach}. In this study we are more interested in sensor based methods and there are already some affordable sensing based-technologies for different application were developed. These types of sensors are called structural-light-based (SLB) sensors \cite{salvi2004pattern}, which is based on the projection of structured light \cite{zhang2012microsoft}. Some of the examples for SLB depth sensors are Kinect by Microsoft, Structure Sensor by Occipital, BlasterX Senz3D by Intel Realsense, Leap Motion Sensors by Leap Motion \cite{marin2014hand} and more are available \cite{mal2018sparse}. These devices have proven to have great impact in these areas but its comes with a trade off with ease of use at consumer application end we will discuss more in section \ref{Chapter4:Dataset}. We will be more focused on depth estimation method based on SLB sensors, more specifically Structure Sensor by Occipital in the entire work.\\

Meanwhile as we know there is a remarkable interest grown in brain style computation and Artificial Neural Network (ANNs) is one of its derivative. ANNs have shown great versatility in the field of detection, classification and prediction. ANNs are used for many applications ranging from image processing \cite{guyon1991applications} , audio signal analysis \cite{bourlard1993continuous}, medical \cite{baxt1990use}, business science \cite{widrow1994neural}, music \cite{nadar2019towards} and many more \cite{zhang2000neural}. It  need a careful processing and considerable domain knowledge for representing a raw data into acceptable form for feature learning process.\note{edit this line please, doesnt make sense,}\nadacn{thanks, but Why? you did not understand? I mean to say ANNs have a large application and you need domain knowledge.. sinal processing is different than image processing.. you get it?} Another import aspect is which comes with brain style computation is hardware limitations towards computational complexity. But in recent days there are significant improvement in the heterogeneous computation \cite{mittal2015survey}. This also comes with difficulties to cope up with low cost and low power platforms for ANNs \cite{mittal2019survey}. \\




\section{Motivation}
\begin{figure}[!b]
    \centering
    \includegraphics[width = 12cm]{Figures/ipad.jpg}
    \caption{Structural SensorImage Scource: Amazon.com}
    \label{fig:Structural_Sensor}
\end{figure}{}

Having a rich scene understanding of indoor spaces can answer various problems related to robotics, human activity recognition etc. Also in some of the immersive experience and interactive technologies like  virtual reality and augmented reality applications where positional tracking and scene knowledge is crucial element of it.

One of our interest is to achieve full reconstruction of indoor environment. This might lead to various multimedia applications like augmented reality games or productive mobile device applications like \textit{IKEA Place} an application developed by Apple in partnership with Ikea. \textit{IKEA Place} helps you to measure and place a 3D model of furniture from the Ikea catalog \cite{lehnert2017neue}. The furniture represents the app as a 3D model in the appropriate scale. You can walk around the piece of furniture with the camera in hand and look at it from all sides. So the applications of 3D reconstruction of indoor environment can have a wide range of usability. As these applications which we are concerned similar to the one which we just described, are more consumer oriented. Therefore we also want to focus at the consumer level usability of such technology in various fields. This is because even though various efficient methods has been proposed based on the projection of structured light (as discussed in introduction and will be discussed in section \ref{Chapter3:RelatedWork} in detail) are available, the application at the consumer level might be challenging with respect to configuration, portability and background knowledge for usability. As we are discussing at consumer level usability, one of the areas we can look into is portable mobile device cameras. As vast majority of cameras which are shipped today are embedded in mobile devices. These cameras have constraints on physical size, mechanical parts and processing capacity and several solutions have been proposed to measure depth. Photographs from portable mobile devices are found to be a new approach for 3D reconstruction besides various traditional sensor based methods \cite{micheletti2015investigating, adan20113d} in other words 3D image reconstruction from 2D images is our focus and motivation to find a solution in this area. To explain better about the consumer level usability as an example we consider the Structure Sensor manufactured by company called Occipital which is as SLB sensor for portable mobile devices like Apple IPads as show in figure \ref{fig:Structural_Sensor}. This comes with an additional hardware installation. Also there have been some efforts made towards low cost and efficient indoor 3D reconstruction using images collected with portable mobile devices or consumer level cameras \cite{ding2019low} but often time these accuracy or quality of these methods always have a trade off with resources like hardware (using SLB sensors, camera quality) or computational efficiency.

\textit{In summary}, as we understand the importance of depth estimation and its possibilities of remarkable applications in various areas, our motivation in this work is 3D reconstruction of a scene from a given 2D RGB image information which will lead to made consumer level application. For this we also want to answer if we can use brain style computational Artificial Neural Network (ANN) methods. because we believe that by implementing a software based method might reduce the complexity at users end but with a trade off of computational power. We are we observe the growth of computational power we believe its not too far to implement a ANNs models in near future. 



\section{Topic Description}
\label{Chapeter1:Topic_Description}
\begin{figure}[h]
    \centering
    \includegraphics[width = 12cm]{Figures/idea.png}
    \caption{The red box of existing system (a) is replaced by green box proposed method (b)}
    \label{fig:Proposed_Model}
\end{figure}{}
%1-2 page(s): What are you doing?
This study is considering neural networks as an algorithm of depth estimation, more specifically using Convolutions Neural Networks(CNNs). The solution should define concise and accurate pixel for 3D reconstruction. Therefore, the goal is to achieve high accuracy in identifying the depth, especially boundaries of closer objects since the application is in indoor environment. Meanwhile, keeping the network operating speeds at a satisfying level, but the final implementation of this network will be on larger processing unit not in mobile device. This could be achieved by choosing a network architecture wisely with an attention to the details of the problem. Monocular depth estimation is based on /note{is this intentional?}ex-ploiting both local properties of texture, gradients, and color, as well as global geometric relations, like relative object placement, and perspective cues \cite{saxena2006learning}. Hence having a global structural understanding of image by Neural Network is very important and at the same time pixel level details for smaller object in a scene. Often time the indoor environment could be more complex than outdoor because of several smaller objects in a scene. For example in office environment, there could be various physical objects on the table where as in outdoor environment having to deal with relatively larger object dimensions are more common.  

Aim of this project is to deliver a robust method and a architecture for depth image prediction replacing the sensor. As shown in figure \ref{fig:Proposed_Model} (a) Existing System for 3D reconstruction relay on hardware based implementation of a SLB sensor which is Structure Sensor figure \ref{fig:Structural_Sensor} integrated with IPad. This is done using Apple AR kit implementation which gives 2D RGB image from IPad and its depth information from the Structure sensor and together are send for further processing of 3D reconstruction. First our primary goal is to replace the existing system highlighted in red box in figure \ref{fig:Proposed_Model} (a) Existing System with a Neural Network as shown in highlighted box in \ref{fig:Proposed_Model}(b) Proposed System. We take advantage of Neural network as this has proven to give great results which we will see in details in section \ref{Chapter3:RelatedWork}.


Secondly, existing solutions to depth estimation from a single image usually rely on assumptions all the far field depth are mapped to a certain threshold which can result as a wall in 3D reconstruction from 2D. For example, the depth range of Kinet v2 sensor range from 0.5 meters (m) to 4.5m, most of work to our knowledge are based on neural network the range above 4.5m are mapped to the maximum value 4.5m \cite{Silberman:ECCV12} which results as a wall. In practical implementation for 3D reconstruction its undesirable for us to have wall. Whereas the output from SLB or Kinet sensors gives a dead pixel (no pixel) or holes for range above any threshold (for Kinet v2 is 4.5m). Another very important reason for having dead pixel is to get perfect reconstruction - ideally we can repaint the dead pixel by changing the position of camera, hence rather than approximation the values around dead pixel and removing dead pixel its important for us to have information about these dead pixel. In our work we are also interested to address this problem. We wanted the the network to learn these dead pixels as true SLB sensors along with the relative depth of the scene for efficient regeneration of 3D scene. 

These two problems gives us a clear direction to formulates some of the specific questions to be answered in this work which is described below. 


\section{Research questions and scientific contributions}
%1 page: Hence, this thesis tries to answer the following research question(s):

This work tries to address the following two questions:
\begin{itemize}
    \item RQ1: Can we Neural Network work as good as SLB sensor in predicting relative depth of a scene?
    \item RQ2: Can a network learn to regenerate the dead pixel from a given 2d image?
    \item RQ3: Can Network learn Different Camera intrinsic parameters?  
\end{itemize}

In order to answer the three questions mentioned above, we propose a state-of-the-art approach using neural networks for this problem. Also generation of a new dataset is required to answer RQ2 and RQ3. This is because most of the approach does not focus in retrieving the dead pixels and existing dataset like NYU Depth dataset \cite{silberman11indoor} Cornell Dataset \cite{3Dscene} , Washington Data V2 \cite{Washington}, Berkeley 3-D Object dataset (B3DO) \cite{Janoch:EECS-2012-85} etc. fail to provide with the dead pixel except NYUv2 dataset with contains raw data information we will discuss more about the dataset in detain in section \ref{Chapter4:Dataset}. To give a jutified reson for RQ3, we need to generate new dataset to have a common ground for comparison. 

Our main two contribution for this work are, first Proposed a neural network model for depth map generation. Secondly we generated new dataset based on Structural (SLB) Sensor.


As we understand that our goal for this work is to achieve a similar working neural network model replacing Structural (SLB) sensor. This might reduce the complexity in users with a trade off of computational power. In this initial phase of research our scope of study how to generate the 3D depth maps from neural networks after training on similar depth data, so all the computation are tested with high processing unit \textit{Nvidia Quadro RTX 8000} graphic card with Graphics processor memory 48 GB GDDR6 with ECCRTX-OPS 84T \nadacn{write about power consumption}. 
And knowing that the final implementation of the Neural Network model will on larger processor but not embedded in mobile device, gives a space of implementing larger and efficient network architecture. 

% Chapter Template

\chapter{Basic Principles}
\label{Chapter2:Background} 

In this chapter we will discuss the basic  principles needed for this work. The entire chapter is divided into two sections. In the first chapter we will go through some theory of neural network and basic building blocks  needed for designing a proper neural network. Most of the concepts related to ANNs are taken from the book by Friedman et al. \cite{friedman2001elements}. Second part will be comprising of some basics of depth estimation and camera parameters concepts. Since the area of Artificial Neural Network is very vast we will be focusing on the concepts which was specifically used for this study. 

\section{Artificial Neural Networks}

As we know Artificial Neural Networks (ANNs) are a developed similar to brain style computation consisting of different level of neurons. In other words we can say its a statistical learning or science of learning from given data. There are various types of learning approaches for solving problems regarding optimization . However, deep learning methodologies have proven to be favorable for many computer vision tasks related classification, detection, prediction etc.\cite{friedman2001elements} 

To understand the basics we first consider the major two types of Neural Network which are supervised learning and Unsupervised learning. In general, we denote \(X\) as input and \(Y\) as output of a given function. For supervised learning, the labels \(y_{i}\) for each input example \(x_{i}\) where    \(\{(x_{1},y_{1}),(x_{2},y_{2}),...(x_{n},y_{n})\} \in X\) for a learning algorithm satisfying function \(f:X \rightarrow Y\) and in such a way that the output \( \hat{y_{i}} = P (y_{i}|x_{i})\) where the output is depended upon the input. All labeled problems can be categorized under supervised learning. For  unsupervised learning, clustering or grouping is basically defined as the detection of similarities and joint density \(Pr(X)\) for a given input examples \({(x_{1},x_{2},...,x_{n})} \in X\) . In supervised learning there is a definite score of success because is \(P(y|y)\) \cite{friedman2001elements}. 

To discuss more about supervised learning with the help of simple binary logistic regression model is given by
\begin{equation} \label{equation:feedforward}
    \hat{y_{i}} = x_{i}W + b 
\end{equation}

In our study we used supervised learning with convolutions network. To have a basic understanding, we discuss few nuts and blots concepts of CNN. The basic building block for a simple CNN is the filter. For the naming convention we use \textit{kernel size} for expressing the size of each filter which will be used throughout this thesis to denote filter size and not to be confused with number of filter. Idea of using filter could be easily understood when related to an example approach for finding edges in an image using a filter designed for edge detection or applying some Gaussian blur. But these filter which are represented in a matrix or in tensor form can be changed, updated or transformed based on \(W\) and \(b\) which is described in equation \ref{equation:feedforward}. Applying these filters in fashion of a sliding window, the network is converting an entire image to a space of the filter which depends on kernel size and also the step of the window slide which is also called as stride. For feature learning methods and filter space adaptation, various attribute contributes to this learning process such as forward propagation and weight \(W\) updates during back propagation which in turn modeled based on various loss function using optimization methods. Optimizer which is basically a minimizing error function. In simple terms it is a error function which helps us to improve our results. Finally in order to tweak the training and optimize the learning progress we have various parameter called hyper-parameters. Some of these parameters are learning rate, in the case for gradient descent approach, decay rate which defined the descent rate after one alliteration over entire training set, various regularizes \cite{friedman2001elements}. 

%where b \in\mathbb{R}



\section{Depth Estimation}
3D spatial awareness have become more and more important with emerging technology of Augmented Reality and Virtual Reality. These 3D scanners use different approaches to estimate the depth out of a scene. Some of the major approaches being used are Stereo Imaging\cite{stereoimaging}, Structured Light System (SLS), Structure from Motion(SfM) \cite{sfm}, Time of Flight(ToF) \cite{timeofflight} and Laser Triangulation \cite{3DLasertechnique}. These approaches either use single frame or multiple frame to generate the third dimension Depth. For example, SfM perceives the depth using motion cues alone where it takes multiple frames and with different point of views of the object to determine the the motion cues\cite{sfm}. This not only needs heavy computation but is time consuming as well. On the other hand, Stereo Imaging computes a Disparity map by matching features of left and right image to calculate the depth of objects with the help of only two images \cite{stereoimaging}.\\
\begin{table}[h]
\begin{tabular}{@{}lll@{}}
\toprule
\textbf{Technique}                    & \textbf{Advantage}           & \textbf{Disadvantage}         \\ \midrule
Structure from Motion          & High capture frequency & Time consuming                 \\
Time of Flight       & No effect of lighting & Low capture frequency     \\
Structured Light System        & High resolution   & Prone to noise                 \\ 
                            &                     &                                               
\end{tabular}
\caption{Comparison of different 3D Scanning Techniques} 
\label{table:3DScanning}
\end{table}
ToF cameras work dynamically through the scene. They scan the environment using illuminating it with incoherent light. To measure the depth, the time taken by the light to reflect back to the sensor is noted. Although, ToF cameras are fast, they are often not accurate. For more accuracy we need more time and hence it only works better with static objects. \cite{tof2} \nocite{Why give a short info and cite}.The comparison between these techniques can also be seen in \ref{table:3DScanning} .These techniques are now widely being used to produce 3D scanners. For example Microsoft Kinect V2.0 uses ToF technique \cite{kinecttof}, while Structure Sensor by Occipital uses SLS \cite{Kalantari}.\\

Structure Sensor(2014) is an open source integration to mobile devices which can capture depth information using Structured Light System(SLS). It consist of a laser-emitting diode, infrared radiation range projector and an infrared sensor to sense the projected radiation. The infrared sensor records the reflecting intensity of the infrared (IR) light pattern projected by the IR projector onto the target while its system on the chip(SOC) triangulates the 3D scene \cite{Kalantari}(As seen in \ref{fig:Structuresensor}). Prime Sense chip (Heindl 2014) is used as the system on the chip(SOC) in Structure Sensor. The hardware can be easily installed to the mobile device through a customized USB 2.0 connection and Structure Software Development Kit. The micro lenses in the projector have different focal lengths which produces a non-uniform pattern of dots which varies at different distances \cite{Kalantari}. \\


\begin{figure}[h]
\centering
    \includegraphics[scale=0.29]{Figures/illustration-of-structure-sensor.png}
    \caption{Illustration of Structure Sensor}
    \label{fig:Structuresensor}
\end{figure}

The biggest advantage of Structure Sensor over any other available depth camera is the size of the hardware. As it is only 120mm long and 28mm wide, One can easily fit Structure Sensor in their pockets. Another advantage of it is that for small rooms and office environments it gives us very accurate results. 
% Chapter Template

\chapter{Related Work} % Main chapter title

\label{Chapter3:RelatedWork} % Change X to a consecutive number; for referencing this chapter elsewhere, use \ref{ChapterX}

%----------------------------------------------------------------------------------------
%	SECTION 1 4+ pages
%----------------------------------------------------------------------------------------
In this chapter we discuss about various recent state of the art approaches and methods for estimating Monocular depths using deep neural networks (DNNs). We will be discussing about different earlier approaches on this subject in section \ref{Chapter3:RelatedWork_EarlyApproach} briefly and look into more specific and elaborated details in recent approaches using Neural Networks based on different important factors in section \ref{Chapter3:RelatedWork_NNModel}. We have also discussed about various datasets which are available. 

\section{Early Approaches}
\label{Chapter3:RelatedWork_EarlyApproach}
Meanwhile understating of structural orientation, recovering range and object depicted in an given image is one of the basic problems of it. Many traditional approaches mainly focus on low-level image cues and geometric structures - most often handcrafted method \cite{torralba2002depth, pentland1987new,lai1992generalized,saxena2006learning}.
 
One of the very early approaches on image understanding with respect to reconstruction of depth image can be traced back till 1982 Barnard et. al. \cite{barnard1982computational} where they discuss various state of the art  computational methods for recovery of depth images at that period of time, which mainly focus on two aspects which are camera geometry and disparity mapping. Later in 1987 Stephen T. Barnard proposed a stochastic approach which provides a dense array of disparities, eliminating the need for interpolation \cite{barnard1987stochastic}. Later D. Scharstein and R. Szeliski \cite{scharstein2002taxonomy} came up with a taxonomy for dense two-frame stereo correspondence algorithms.\\

\subsection{Probabilistic Models}
\label{Chapter3:RelatedWork_ProbabilisticModel}
Probabilistic Models were the initial steps towards DNNs. Sexana et al. \cite{saxena2006learning} in 2006 were one of the first to propose an probabilistic model to predict depth image from single image. This approach was based on Markov Random Field (MRF) using Gaussian and Laplasian distribution model. Features where considered on small patch level of a given image \footnote{images were sub divided into small patches} in two distinctive levels. First, to estimate the absolute depth and second to estimate relative depth. This relative depth were calculated based on magnitude of the difference in depth between two patches for which 3 different local cues such as texture variations, texture gradients and haze were considered. Distance 3D scanner where used for data collection for this purpose. Their work was extended to 3D scene reconstruction with improved MRF model, Make3D \cite{saxena2008make3d} system for 3D model generation. One of the challenges of this system is that the images relies on horizontal allied calibration. 

This work lead to various new probabilistic models which in recent years can be classified into Deep Neural Networks (DNNs). Recent years many solutions has be tried for monocular depth estimation problem such as supervised, semi-supervised and unsupervised learning.   

One of the inspiring modern approach by Eigen et al. 2014 \cite{eigen2014depth} where two network component stack were used namely global coarse-scale network followed by local fine-scale network. Global coarse-scale network predict overall depth structure which is intern then refined by a local fine-scale network. This opened doors for fusing feature maps from different level for better predictions.
Also many Convolution Neural Networks (CNNs) based models where used to understand the relationship between RGB images and its corresponding depth maps \cite{liu2015deep,laina2016deeper,Eigen_2015_ICCV,eigen2014depth, Alhashim2018}. By this time encoder-decoder style of architecture where famous\cite{Alhashim2018, hu2019revisiting} which we will discuss in detail in section \ref{Chapter3:RelatedWork_NNModel}. 

Meanwhile some other works focused of various other details like,  Ladicky et al. \cite{ladicky2014pulling} highlights the limitation of the various data driven approaches for monocular depth estimation by exploring the structural perspective  geometry. Zammir et al. \cite{zamir2018taskonomy} studied on the modeling structure space of visual task and investigating on transfer learning dependencies, one of the finding where, the demand of labeled dataset could be reduced by introducing transfer learning approaches since there is always model and structural dependencies. Ha et al. \cite{ha2016high} also proposed a high quality depth map from non calibrated short video clip. Shu et al. \cite{Shi2015BreakAR} proposed that small-scale de focus blur can enhance the depth prediction and Fouhey et al. \cite{Fouhey_2013_ICCV} also tried to learn structural component.




\section{Recent Neural network Approach}
\label{Chapter3:RelatedWork_NNModel}
 Remarkable advances has been made in deep learning Laina et al. \cite{laina2016deeper} proposed a new architecture build on ResNet-50 by replacing the last layer with up sampling layers for reconstructions. They also proposed new loss function \textit{Huber Loss} an end to end approach, model can learn geometrical relationship. In the same way some approaches replaced ResNet backbone by different pretrained models as encoder\cite{Alhashim2018, hu2019revisiting}. More recent work focused on combining information from multiple scale, encoder and decoder style. This is to get the different level learned features \cite{Xu_2018_CVPR, Eigen_2015_ICCV} then concatenated at the decoder part of up sampling stage of architecture. One of main reason for such approach is to get higher spatial resolution by eliminating distorted and blurry edge since the probabilistic distribution always results into smooths object boundaries\cite{hu2019revisiting}. Also having features learned from top layers contains higher level information like which can give a global understanding of structural aspect of a image or scene. Meanwhile there where also some unsupervised learning methods applied for the same task \cite{godard2017unsupervised, qi2018geonet}.

We have analyzed the resent work in detail and will be describing more about different approaches categorizing into three groups model architecture, input representation and dataset used. These are the most commonly known to be the most important factors which defines the performance of a Network.

\subsection{Model Architecture}
\label{Chapter3:ModelArch}
In recent years since many encoder-decoder and multi scale style proven to give better results \cite{Alhashim2018, hu2019revisiting}. In this work we have used the similar architecture. In these encoder-decoder architecture the encoder always comprises of a backbone of a larger pretrained model. The most common encoder backbone which could be found are Residual Network (ResNet), Densely Connected Convolutional Network (DenseNet), Squeeze and Excitation Network (SENet) or Visual Geometry Group Network (VGGNet)\cite{hu2019revisiting}. Very deep neural network are difficult to train due to vanishing and exploding gradient. ResNet helps to skip intermediate identity connections by 
\begin{equation} \label{eqResNet}
    {a^{i+n}=g(z^{i+n} + a^i)}
\end{equation}


where \(g\) is the non-linear activation (eg. ReLu) and \(z\) is the output of linear activation (or output of a particular layer) of $i^{\text{th}}$ layer and \(a\) denotes the output of a layer. In contrast, DenseNet is an extension of ResNet where instead of skipping the and merging with the $i+n^{\text{th}}$ layer as an addition, DenseNet performs concatenation of all the $n$ skipped feature maps
\begin{equation} \label{eqDenseNet}
    {a^{i+n}=g(z^{i+n} + a^{C})}
\end{equation}


where \({C =C_i([i,i_2,...,i_{n-1}])}\) denotes the concatenation of layers till previous layer \(i_{n-1}^{th}\)  using the summation \cite{huang2017densely} there by helping th feature propagation, feature reuse and  reduce the parameters. Where as SENet is a transforms a set of block to another. It comprises of \textit{squeeze} operator and \textit{excitation} operator.\textit{squeeze} operator aggregates the feature map and \textit{excitation} operator  aggregates the learned activations Net \cite{iandola2016squeezenet}. 


In table \ref{table:RelatedWork_STA_Architecture} based on the various neural network model architecture we compare with recent 6 different architecture which are mentioned. These models are selected based on the most recent approaches from past one year and sorted based on the Root Mean Squared Error (RMSE). All the RMSE score are based on test set of NYU v2 depth map dataset \cite{silberman11indoor}.

\textbf{M1}, \textbf{M3} and partially \textbf{M4}, were build upon ResNet backbone for depth prediction. Even though all the three methods are different, \textbf{M3} and \textbf{M4} methods have one thing in common which is that both architecture where designed for multi-task specific model after encoder-decoder part. \textbf{M3} Pattern Affinitive Propagation (PAP-Depth) method idea was developed based on the affinity behaviour between two similar task \footnote{for example finding Surface normal and depth are related task, These two factors remains the  most important component for image segmentation}. This can be described by Affinity Block or sub network, affinity block comes after the up-sampling (decoder part) the last layers are fed into three different task-specific networks for prediction of Depth, surface normal and segmentation. They also integrate multi-scale features derived from different layers of encoder with each task-specific network. Each task-specific network has two residual blocks, and produces the initial prediction after a convolution layer. Then conduct cross-task propagation to learn the task-level affinitive patterns. Each task-specific network firstly learns an affinity matrix by the affinity learning layer to capture the pair-wise similarities for each task.

\textbf{M3}, SharpNet basically was built to after encoder-decoder style for addressing pixel-perfect near occluding contours  problem. The approach of SharpNet is very similar \textbf{M4}. Instead of segmentation sub-net SharpNet has  occluding contours as on of the 3 multi-task network model after decoder part. The features from encoder in integrated to this sub-network. 

\textbf{M1}, Geometric Network (GeoNet) is a two layer architechure - built upon two two big CNN model ResNet and VGG in two different block. First block is two models (ResNet and VGG) are trained separately. VGG is used for depth images and ResNet for surface normal prediction of a image. The output is given to second block which consiste of two sub network namely \textit{Depth-to-Normal} and \textit{Normal-to-Depth}. As the names States the Depth output from VGG is fed to  \textit{Depth-to-Normal} to get refined depth and vice version to get refined Normal

\textbf{M2} and \textbf{M6} are built upon DenseNet backbone. In \textbf{M6} one of the distinctive idea is, the decoder part has multiple resolution block. If \(\textbf{\textit{$B_n$}}\) is denoted as one block of decoder to obtain output resolution of \(\textbf{\textit{$D_n$}}\) , Then  \(\textbf{\textit{$D$}}\) depends on number of  \(\textbf{\textit{B}}\). In this approach they have 5 different resolution \(\textbf{\textit{$D_n$}}\)  such that \(n = 5\) for extraction of multi resolution depth from different positions to get relative depths. Final depth image is obtained from relative pair-vise comparison with the relative depth images from different \(\textbf{\textit{$B_n$}}\)

\textbf{M2} DenseDepth is a much simpler model than all the above approaches yet proven to have best results than others in terms of RMSE on NYU dataset. There are two distictive  ideas. First, they use multi-scale encoder featured concatenated with decoder part to get a better strutural higher level features, also the decoder is made of \(2 x \) bilinear up sampling method. Secound, they use transfer learning approach.  

\begin{table}[t]
\centering
\begin{tabular}{p{0.05\linewidth}p{0.2\linewidth}p{0.1\linewidth}p{0.3\linewidth}p{0.2\linewidth}}
%{|c|c|c|c|c|}


\hline
\textbf{\# } & \textbf{Method} & \textbf{RMSE} & \textbf{Backend}& \textbf{Year/Reference} \\ \hline\hline
M1              & GeoNet          & 0.445             & ResNet and VGG  & 2018 \cite{qi2018geonet}          \\ \hline
M2              & DenseDepth      & 0.465             & DenseNet        & 2018 \cite{Alhashim2018}         \\ \hline
M3              & SharpNet        & 0.496             & ResNet       & 2019 \cite{ramamonjisoa2019sharpnet}\\ \hline
M4              & PAP-Depth       & 0.497             & ResNet          & 2019 \cite{Zhang_2019_CVPR}         \\ \hline
M5              & SENet-154       & 0.530             & SENet           & 2018 \cite{hu2019revisiting}          \\ \hline
M6              & RelativeDepth   & 0.538             & DenseNet        & 2019 \cite{lee2019monocular}          \\ \hline
\end{tabular}

\caption{Investigated neural network architecture on Monocular Depth Estimation on NYUv2 Dataset}
\label{table:RelatedWork_STA_Architecture}

\end{table}


There are some advantages of using DenseNet over ResNets. One of the first advantages is there is a strong gradient flow because increase in the depth of a CNN might result in vanishing gradient problem . Second, we get more diversified feature, which means there can be good generalized information from the previous layers which tends to have richer patterns where as in skip style ResNet such information are lost. 

In summary, despite of having simpler model by Alhashim et. al \cite{Alhashim2018}  the results seems convinsing when compaired with various other complex and big models. Having found the most common encoder backend for encoder-decoder style nerwork are  ResNet and DenseNet. These both have proven to to have state-of-art results. Inspired by the simplicity and the results of  Alhashim et. al. work on DenseDepth we use DenseNet model as out backbone encoder in our work for estimating depth maps. We would like to highlight three distinctive reason for selecting thing approach. First as we see DenseNet have proven to have some advantages over for deeper network. Second the model uses transfer learning and trained 120K images from NYU v2 datased which weights have be shared to public Which reduces the computation needed for re training this architecture again. Third, many recent models have verious multi-sub task network for their work, in our case we are more focused in structure sensor as described in section \ref{Chapeter1:Topic_Description} 


\subsection{Input Representation}
One of the important aspect of feature based learning is input representation of a feature to its respective model \cite{friedman2001elements}. The most common approach of input representation is in its simple form of RGB and depth pair from various SLB sensors \cite{eigen2014depth, xu2017multi}. Some approaches also exploits the structural orientation by computing surface normal \cite{li2015depth, qi2018geonet}. One of the approach Zhang et. al. \cite{Zhang_2019_CVPR} was to utilize three features  depth, surface normal and segmentation fused together by concatenation and given as input at encoder and at the decoder part a multi-sub task specific network was designed. The advantage of such multi feature is it can have a good cross task learning - which means different feature representation for same temporal content. One more example of  cross task learning is \cite{qi2018geonet} where two model where used for two different input feature depth image and surface normal. Another type of input which was used in some approaches is sparsity  \cite{chen2018estimating, mal2018sparse}. Chen et. al. \cite{chen2018estimating} computed the sparsity matrix and given as addition input along with depth maps. In our work we used the most common approach RGB - depth, one of the map pair as input feature.  

\subsection{Existing Datasets}

By now we have a understanding of various Neural Network methods used to approach this study of depth estimation from previous Sections. One of the most important component in feature based learning methods are highly dependent on data itself \cite{friedman2001elements}, which mean the result of such probabilistic output models can be highly influenced by the dataset itself. For example, We can either fit a model by training with the small amount of dataset to perform few particular designated task in a designed environment or we can train a model on a large and diverse database to perform more generalized tasks.

There are several datasets readily available for different purposes. One of the major classification of dataset for depth maps can be classified as either by the environment scene - indoor and outdoor environment or based on application, for example KITTI for Autonomous driving \cite{Geiger2013IJRR} scenario and Berkeley 3-D Object Dataset (B3DO) \cite{Janoch:EECS-2012-85} is soulely focused for Object Recognition. Here we categorize on the basis of the process of producing the dataset.

Saxena et al. \cite{saxena2006learning} proposed a dataset of 959 Images of RGB and laser range data. All the depth data was collected using a custom-built 3D laser scanner. The images are are of dimensions 2272 $\times$ 1704, while the depth maps are 55 $\times$ 305. Later, Silberman et al. \cite{Silberman:ECCV12} proposed a high quality Kinect dataset (NYU Depth-V2) in 2012 which is now being used widely across the in 2012. NYU Depth-V2 \cite{Silberman:ECCV12} consist of 1449 densely labeled pairs of aligned RGB and Depth images. By labeled pair we refer to segmentation of all the surfaces with respected surface normal. Furthermore, it consists 407024 unlabeled frames. Former NYU dataset(NYU Depth V1) \cite{silberman11indoor} consisted of only 67 scenes while NYU Depth-V2 consist of 464 different indoor scenes. Both the images and Depth maps are 640 $\times$ 480 in resolution. The dataset is appreciated for the segmentation of a room.

Geiger et al. \cite{Geiger2013IJRR} in 2013, proposed an outdoor dataset consisting of stereo depth images as The KITTI Dataset. KITTI Dataset is used for Autonomous Driving and Robotics purpose. It includes high resolution color and grayscale stereo camera images, laser scans, high-precision Global Positioning System(GPS) measurements and SLAM data. The main intend was to push forward the development of computer vision and robotic algorithms targeted to autonomous driving.

In 2016, Mayer et al. \cite{MIFDB16} produced three synthetic datasets called Scene Flow Dataset providing over 35000 stereo frames with dense ground truth for optical flow, disparity and disparity change, as well as other data such as object segmentation. The resolution of these images is 960 $\times$ 540. MPI Sintel \cite{Butler:ECCV:2012} is also a synthetic depth dataset which is available online. All these dataset were created using the open source 3D creation suite Blender.

Another traditional approach to collect dataset is using crowd-sourcing. Chen et al. \cite{DBLP:journals/corr/ChenFYD16} introduced "Depth in the Wild" dataset in which they took images from Flickr \footnote{ \url{www.flickr.com}}  and presented the crowd-sourced workers with these images with two highlighted points asking which of the point was closer. The dataset consists of 495,000 diverse images, each annotated with randomly sampled points and their relative depth. 

\begin{table}[]
\begin{tabular}{lllllll}
\hline
{\textbf{Dataset}} & {\textbf{Approach Used}} & \textbf{Environment}        & \textbf{RGB resolution}           & \textbf{Depth resolution} & \textbf{Number of images} & \textbf{References}                                  \\ \hline
Make3D                                & Custom Laser Scanner                      & Outdoor                     & 2272 $\times$ 1704                & 55 $\times$ 305           & 959                       & http://make3d.cs.cornell.edu/data.html               \\ \hline
NYU V2                                 & Kinect V2                                   & Indoor                      & 640 $\times$ 480                  & 512 $\times$ 424          & 407,024                   & \textbackslash{}cite\{Silberman:ECCV12\}             \\ \hline
KITTI                                  & Stereo Imaging                              & Outdoor(Autonomous Driving) & 1392 $\times$ 512 (Original Size) & -                         & 15,000                    & \textbackslash{}cite\{Geiger2013IJRR\}               \\ \hline
B3DO                                   & Kinect V2                                   & Indoor                      & 640 $\times$ 480                  & 512 $\times$ 424          & 849                       & \textbackslash{}cite\{Janoch:EECS-2012-85\}          \\ \hline
Cornell Dataset                        & Kinect                                      & Indoor(Robotics)            & 640 $\times$ 480                  & 640 $\times$ 480          & 550                       & \textbackslash{}cite\{3Dscene\}                      \\ \hline
Washington V2                          & Kinect                                      & Indoor(Object Recognition)  & 640 $\times$ 480                  & 640 $\times$ 480          & 300                       & \textbackslash{}cite\{Washington\}                   \\ \hline
Scene Flow Dataset                     & Synthetic/Stereo                            & Outdoor                     & 960 $\times$ 540                  & 960 $\times$ 540          & 39000                     & \textbackslash{}cite\{MIFDB16\}                      \\ \hline
Depth in the Wild                      & Crowd Sourcing                              & -                           & -                                 & -                         & 495,000                   & \textbackslash{}cite\{DBLP:journals/corr/ChenFYD16\} \\ \hline
\end{tabular}
\caption{Comparision of various Datasets}
\label{table:DatasetComparision}
\end{table}

There are many Other datasets are available like Cornell Dataset \cite{3Dscene} , Washington Data V2 \cite{Washington} and  Berkeley 3-D Object dataset (B3DO) \cite{Janoch:EECS-2012-85} also follows the same approach of labelling the environment either for object recognition or robotics purpose. All of them were also captured using Kinect camera approach producing RGB-D images. We can see the comparison of all the datasets in table \ref{table:DatasetComparision}.

As we can see, all of the Datasets mentioned above are suitable for different application scenarios. For example, NYU V2 \cite{Silberman:ECCV12} is best for Segmentation purpose, B3DO \cite{Janoch:EECS-2012-85} is suitable for Object Recognition and KITTI Dataset \cite{Geiger2013IJRR} for Autonomous Driving purpose. Keeping in mind the scope of the research is to estimate depth of monocular indoor scenarios and office environments, NYU V2 is the most relevant as it has high quality depth maps for indoor environment which can be exploited for training of the network. As discussed in section \ref{Chapter3:ModelArch}, we produce a small dataset using a Structure Sensor to make the model more focused towards mobile devices which is discussed in next chapter.



%\section{Additional Details}
%\url{https://paperswithcode.com/sota/monocular-depth-estimation-on-nyu-depth-v2?p=high-quality-monocular-depth-estimation-via}%
%upsampling cound be in many parts

%\begin{itemize}
 %  \item Nearest Neighbor
  % \item Bilinear Interpolation,  A single pixel value is calculated as the weighted avg.
   %\item Transposed Convolution, we have weights that we learn through back propagation.
%   \item Recent works
%\end{itemize}


%Supervised 
%CNN - slam - \cite{Tateno_2017_CVPR}
%CRNN
%GeoNet

%\\
%unsupervised \\
%un supersed CNN - geometry  - \cite{garg2016unsupervised}
% Unsupervised Learning of Depth and Ego-Motion From Video- \cite{Xu_2018_CVPR}
%Left and right

%\\
%Transfer learning\\
%CNN -

%\\
%{https://github.com/GabrielMajeri/nyuv2-python-toolbox}\\
%{https://github.com/ayanc/mdepth}\\
%Some \cite{bhoi2019monocular} \cite{laina2016deeper, bhoi2019monocular}\\
%We use this curretn state of the art 

%
% Chapter Template

\chapter{Structure Depth Dataset}

\label{Chapter4:Dataset} 

%----------------------------------------------------------------------------------------
%----------------------------------------------------------------------------------------
%\url{https://fenix.tecnico.ulisboa.pt/downloadFile/1689244997256744/Thesis.pdf}

In this Chapter, we present our dataset created using Structure Sensor and we call it as \textbf{Structure Depth} dataset. The entire chapter is divided in to three sections. In the first section we discuss the technical details of the Structure Sensor and followed by the methods involved in creating this dataset. The last part of this chapter is about the various pre processing methods we used to performed for specific experimental setups.


\section{Structure Depth Dataset Overview} 

Our aim in this work is to deliver a robust system for depth prediction as seen in the Figure \ref{fig:Proposed_Model} for a portable hand held mobile device. In our case it is an IPad integrated with Structure sensor. In order minimize the difference between the predicted depth maps and depth map generated from Structure sensor we need to train with the similar dataset. We have also validated this in our experiments which will be discussed below in Section \ref{Chapter6:Results}. As discussed in the Section \ref{Chapeter1:Topic_Description}, this will also help us to study the effect of different camera and sensor properties on different environment.\\

As we have seen earlier in Chapter \ref{Chapter3:RelatedWork}, there are multiple datasets available but almost none fit our context. As we are already exploiting the depth features from NYU V2 for general prediction, we would like to produce a dataset which is solely application based. Here our application is mobile device, particularly IOS, we generate the dataset using an IPad. This will not only make the intrinsic parameters to be learned by the network but also particularize for IOS based devices. As we use IPad as integration, all the RGB Images are received from the IPad's camera. The features we want the neural network to learn and predict should have the same camera intrinsic parameters in order to decrease the amount of pre processing of the dataset. Hence, We first calibrate IPad with Structure Sensor before collecting the Dataset. Another major advantage of using Structure Sensor with an IPad over Kinect is the resolution of RGB and Depth Images. While Kinect V2 has $512\times424$ Infrared camera resolution, Structure Sensor can go upto $640\times480$. \\

Our Dataset consist of 2 versions. The first version(SD\_V1) which was taken during the training of \textbf{A1} network which is detailed in Section \ref{Chapter5:Methodology}. In V1, we produced no hole depth images with wall of threshold 4 m. Since the temporal resolution of Structure Sensor is as high as 60 fps, there is not much difference in frames. So, in order to achieve a quality dataset, we saved a frame in every 10\textsuperscript{th} frame. Having a 60 fps video stream, we saved 6 fps. This removes redundant data and prevents us from overfitting. In the second version(V2) of our dataset, we captured every 2\textsuperscript{nd} frame per second giving us a temporal resolution of 30 fps. In V2, we removed the distance threshold and filled it with holes instead. We trained \textbf{A2} network with the new dataset. Including V2, we have a total of 18 scenes and consisting of 2675 images captured using an Apple iPad Pro version 12.9 (2015) model and a Structure Sensor only. The dataset majorly includes images from office and classrooms environments. It consist of RGB Images of $640\times480$ as 3 channel $\times$ 8 bit int and Raw Depth Images of 640 $\times$ 480 as 1 channel. Both of the images are saved using lossless compression. Later, we also perform some processing to synchronize RGB and depth images and to fill the holes of depth image which will be discussed later in this chapter. We also provide the camera extrinsic parameters as a numpy matrix of $4\times4$ which is useful for reconstruction of a scenario.\\


\subsection{Technical Specification of Structure Sensor}
The Structure Sensor \footnote{\url{https://structure.io/}} is a 3D scanner introduced by Occipital in 2014. As the name suggests, it uses Structured-Light-System(SLS) 3D scanning\cite{Kalantari}. It can be easily attached to an IPad and with Structure SDK it enables us to generate good quality RGB image stream from IPad and its realtive depth stream from Structure Sensor at the simultaneously. As it only weighs 95 grams and can be used as an extension to mobile device, it makes the sensor very portable. With possibility of recording 60 frames per second at a resolution of $640\times480$\cite{Kalantari}. When compared to Kinect V2, it has more depth image resolution. The comparison can be seen in table \ref{table:KinectVsStructureSensor}. The minimum range it can capture is 40 cm while it can capture till 4 m with fine precision. After 4 m the precision is not satisfactory enough to use it for the dataset. Occipital claims it has low frame-to-frame noise and provides 100\% fill rate on most of the materials. On the other hand IPad Pro 12.9 provides us with an RGB video stream of $1920\times1080$ at 30 frames per second and $1080\times720$ at 60 frames per second. IPad Pro also provides us with Camera extrinsic parameters which are useful for the reconstruction of the scenario.\\

\begin{table}[h]
\begin{tabular}{@{}lll@{}}
\toprule
\textbf{Features}                    & \textbf{Kinect V2}           & \textbf{Structure Sensor}         \\ \midrule
Depth Sensor Type           & ToF & SLS                 \\
RGB Camera Resolution       & $1920\times1080$, 30 fps & $1920\times1080$, 60 fps     \\
IR Camera Resolution        & $512\times424$, 30 fps   & $640\times480$, 60 fps                 \\ 
Field of View of IR Camera  & $70^\circ\times60^\circ$           & $58^\circ\times45^\circ$                         \\
Recommended Operative Range & 0.4 m - 3.5 m       & 0.4 m - 3.5 m                                  \\
           &                     & 
\end{tabular}
\caption{Comparison of Kinect V2 and Structure Sensor}
\label{table:KinectVsStructureSensor}
\end{table}
\begin{figure}[h]
    \centering
    \includegraphics[scale=0.4]{Figures/holes.png}
    \caption{Generation of holes due to parallax}
    \label{fig:holes}
\end{figure}
While these sensors are great devices they have some limitations. The distance they can measure is limited and they suffer from reflection problems on transparent, shiny or very matte and absorbing objects. Another limitation is holes generated by parallax effect happening due to difference in position of the camera of IPad and Structure Sensor\cite{Kalantari}. In simpler terms, it works like human eyes. If we look at an object using only either of our eyes, we could notice disparities. When there is an object we try to focus on, our eyes which located at different position like the sensors in the Structure Sensor, they will see the object from different angles. If the object is close enough, then sometimes one eye can see what is behind and other can not. This can be seen in figure \ref{fig:holes}. As a result, it produces a shadow of holes which can be seen in figure \ref{fig:holes2}.



\begin{figure}[h]
    \includegraphics[scale=0.29]{Figures/RGB.png} \includegraphics[scale=0.37]{Figures/Depth.png}
    \caption{Holes produced in depth image}
    \label{fig:holes2}
\end{figure}


\subsection{Dataset Collection}

A significant role in Machine Learning is played by Dataset and the collection of it makes most influence on the features that network learns. As the distance is limited in Structure Sensor and due to our scope of research we focus on indoor offices and classrooms environments. While capturing the dataset, one should keep in mind that the network should learn only the novel features not the artifacts. One good instance of this would be capturing the depth of a screen. Since, a screen has reflective black surface, it loses depth information resulting holes in depth image as seen in Figure \ref{fig:screens}. If we feed these images to the network, it might learn the features such as, screen is always at, let say x distance, where x is pixel value we assign to such holes. These holes are the undesired features and called artifacts. Such artifacts could be produced due to various reasons. It could be the reflective/absorbing nature of the surface of an object or the distance and position of the object\cite{geomar41830}. As a reflective/glossy surface leads us to wrong or no generation of depth pixel, we try to avoid them while capturing. Figure \ref{fig:screens} is one good example of such holes. As we notice in the RGB image there are two screens, one is predicted in some regions while other one is totally lost. In indoor environments, such objects are inevitable. Thus, in such problems we use techniques like inpainting \cite{inpainting}. Image inpainting could also be used to resolve artifacts due to parallax effect discussed in section previous section.\\
\begin{figure}[h]
    \includegraphics[scale=0.29]{Figures/7.png} \includegraphics[scale=0.24]{Figures/Raw7.png}
    \caption{Holes produced in depth image due to reflectivity of the screen}
    \label{fig:screens}
\end{figure}
 

Since, the accuracy decreases with the increase in depth \cite{deptherror} and because the recommended range is no more than 4 metres(m) for Structure Sensor \cite{Kalantari}, we remove all the pixels which are far away than 4 m. These pixels can be resolved using further predictions and post processing. This also protects our network from learning the artifact of distant objects as a wall as discussed in \ref{Chapeter1:Topic_Description}.\\

\subsection{Data Processing}

The very first step of processing the dataset would be registering the depth pixels. After we have calibrated the camera, we convert the pixels received from Structure Sensor to millimetres(mm) for ease of understanding as the pixel value have a non linear relationship. After we have the distances in metres, we want to eliminate all the possible artifacts as these images still contain holes and precision less pixels. Usually, far away pixels have less precision and results in holes as well\cite{deptherror}. It is a good idea to conserve those pixels as holes in order to reconstruct the environment by later predicting the future frames. Firstly, after we have the Depth images in metres, we save all the holes by its pixel position or index for all the pixels more than 4000 mm so that we can use that information later. By doing so, we can eliminate the distance threshold artifact discussed earlier.

\begin{figure}[h]
    \centering
    \includegraphics[scale=0.35]{Figures/versions.png}
    \caption{Versions of our proposed dataset}
    \label{fig:datasetversion}
\end{figure}

\begin{figure}[h]
    \includegraphics[scale=0.50]{Figures/process.png}
    \caption{Different stages of processing the depth frames}
    \label{fig:processing}
\end{figure}

As discussed earlier, the Depth images contains holes from shiny/glossy objects\cite{shiny}. In office environments, these holes are mostly generated by Computer Displays. Since we do not want our network to learn that all the screens are close(0 pixel), we interpolate these holes. We do this by inpainting \cite{inpainting} the whole image. This fixes the issues of shadows produced due to the parallax as well. Noting that it inpaints the pixels more than 4 m too. V1 of our dataset consisted of depth images which were totally inpainted and there were no holes as can be seen in \ref{fig:datasetversion}. Basically which means, if some of the pixels which are far away than 4 m are missing, they are interpolated using nearest neighbours. As we know, these predictions are not very accurate from past discussion, it is also possible that the inpainted depth is not correct. So In V2, successor of V1, we use the index of pixels of depth more than 4 m we saved now later in the process. This will reproduce all the holes which were more than 4 m as missing information is better than wrong predictions. These holes can later be predicted using the future frames. Our depth image is now ready to be fed to the network but as one would have noticed before, The size of the Color Image and Depth Image does not match since they have different aspect ratios due to different camera sources. So, In the end we crop the depth image centre focused in order to make them compatible with the network. We can understand this through Figure \ref{fig:processing}, where we show how the captured data and is processed and stored. \\






\newpage 

\chapter{Methodology}

\label{Chapter5:Methodology} 
%----------------------------------------------------------------------------------------
%	SECTION 1
%----------------------------------------------------------------------------------------

In this chapter we will discuss the entire work flow and the methods used in this study. Starting from the different model architecture used. We used two two main architecture and with some configurations to answer our research questions. Also we will be discussing how the structural dataset was created and some pre processing applied in order to investigate on research question.


\section{System Overview}
We approached this problem of depth map estimation from monocular images to give end to end solution as much as possible using two CNN. As we discussed in the section \ref{Chapeter1:Topic_Description} our main focus was (a) to deliver a robust method and (b) regeneration of dead pixels or holes for better reconstruction of images. For this we consider two CNN models, first, we propose a model and this approach we call it as \textbf{A1} and the second model approach is a state of the art model and we call it as \textbf{A2}. To deliver a state of the art method and results we try to propose \textbf{A1} and validate against \textbf{A2}. In this section we briefly describe both our models and input representation used for the experiments. And further more in order to improve our results we also propose different configuration methods. These configuration methods and motivations are described in details in the following section \ref{Chapter5:Experimental_Setup}. 


\subsection{Input Representation}
For our study we use two datasets NYU\_v2 Depth \cite{silberman11indoor} whose depth information was obtained from kinet sensor and the other dataset was created as a part of this study from Structure sensor which is described in chapter \ref{Chapter4:Dataset}. The input dimension of target RGB images for both the approaches \textbf{A1} and \textbf{A2} are (480$\times$640$\times$3). Whereas input dimensions of ground truth depth image for \textbf{A1} is same as the target image  which is (480$\times$640$\times$1) while for \textbf{A2} is  (240$\times$320$\times$1) which is half of its target input resolution.

There where two pre- processing methods applied on Structure sensor one to train on holes and other with nearest interpolation method. Different input representation where used for different configuration which are described in details in the following section \ref{Chapter5:Experimental_Setup}.

\subsection{Approach 1 (A1) - U-Net Style Network}

\begin{figure}[t]
    \centering
    \includegraphics[width = 14cm, height = 9cm]{Figures/A1.png}
    \caption{U-Net architecture (\textbf{A1}). The arrow mark shows the concatenation of the higher level learnt feature from encoder to the upsamplng decoder part.}
    \label{fig:A1-U-NetArchetecture}
\end{figure}{}

U-Net (\textbf{A1}) architecture as shown in Fig \ref{fig:A1-U-NetArchetecture} is built upon simple idea of using upsampling layer or de-convolution layer  \footnote{not to be confused with the nomenclature, there are various names for de-convolution such as up-sampling layers or transposed convolution layer used by tensorflow and keras (\url{www.tensorflow.org/})} for up scaling the learnt feature and regeneration of depth maps. Hence the decoder block comprises of transposed 2D convolution layers which have the same dimension as the kernal size of encoder block. This means number of layers in encoder and decoder are the sample that's why the name U-Net. We use \textbf{A1} architecture since its easy to configure and understand how the up sample of network can be done. Another advantage is training such a small network is faster but it is a challenge is the network can learn the low level features. 

\textbf{A1} consists of 4 blocks of encoder and decoder part. Each block of encoder consists of 4 layers, 3 layers of 2D Convolutions and 2D max pooling layer of size 2$\times$2 at the end of each block. We designed the 3 convolutions layer with the increasing order of number of filter 4, 16 and 32 with kernel size of 3$\times$3. In the similar way in each decoder block we have 4 layers, starting with a up-sampling layer with kernel size if 2$\times$2 which means the up scaling of the image is in the factor of 2. Nearest neighbor interpolation method used by the up-sampling layer \footnote{\url{www.tensorflow.org/api_docs/python/tf/keras/layers/UpSampling2D}} implemented by keras layer. Towards the end each decoder block as 3 convolution 2D stacked in the similar way as encoder blocks. Each decoder block is concatenated with relative encoder block with has same dimension as shown in fig. \ref{fig:A1-U-NetArchetecture} with arrow marks. The dimension of encoder and decoder block output tensors are kept in symmetrical. This can be achieved by having same kernel size of max pooling at encoder part and up sampling layer at decoder, hence the symmetry in the emcoder and decoder layers. The last output layer comprises of Convolution 2D layer of 1$\times$1 with sigmoid activation. Also last layer has no stride to keep the dimension same as the input. 
This model has input dimension of (batch size$\times$480$\times$640$\times$3) and output dimension of (480$\times$640$\times$1) for both target RGB image and ground truth depth images.



\subsection{Approach 2 (A2) - DenseNet backbone}

\begin{figure}[h]
    \centering
    \includegraphics[width = 15cm,  height = 10cm]{Figures/A2.png}
    \caption{\textbf{A2} with DenseNet backbone}
    \label{fig:A2-DenseNet-arch}
\end{figure}{}




As shown in fig. \ref{fig:A2-DenseNet-arch} shows an overview of our second approach \textbf{A2} for depth estimation. We adapted the idea proposed by Alhashim et. al. \cite{Alhashim2018}. The basic idea of this approach \textbf{A2} was to use DenseNet as backbone or as an encoder from our literature study on this topic. The input RGB image is fed to the DenseNet-169 \cite{huang2017densely} network which is pretrained on ImageNet \cite{deng2009imagenet}.  Backbone DenseNet architecture is designed in a feed forward fashion within a dense block or in other words for feed forward design is that each layer is directly connected to every other layer. But since they adapt DenseNet style there is also skip connections between the layers. ImageNet is a large database of images which was build upon WorldNet which is organized in a hierarchical manner, which means the images are trained and clustered according to carious classes which give a good hierarchical tree and sub-tree format for classification or clustering. Another great advantage of this net is its versatility of the classes ranging from mammals, vehicles, birds to furniture with 12 subtree which gives us 5247 category at the time of this paper by Deng et. al.\ref{deng2009imagenet}. ImageNet claims to have a order of 50 million images and an average of 5000 image per node \footnote{\url{http://www.image-net.org/}}. ImageNet is also good for object recognition, classification and also clustering problems. Due to such versatility ImageNet can give a us a good generalisation of the physical structure from a 2D image which is very important for our task. This also has proven to give state of the results by the model proposed by Alhashim et. al. \cite{Alhashim2018}. 


The design of this network is in such a way that the output from the DenseNet block is  fed to a successive series of up-sampling layers. The decoder has 5 Bilinear up-sampling blocks. \(4^{th}\) block is designed in a way that it give the output resolution half of its input, which is 230$\times$320$\times$1 for input size of RGB image 460$\times$640$\times$3. The \(5^{th}\) block has up-sampling factor of 2 which gives us full resolution same as input. In all our experiments we use only 4 blocks and in post processing we up sample by the factor 2 to get full resolution. This saves some computational time. The decoder part is trained on the 120,000 images of NYU v2 depth dataset. Also the decoder does not contain any Batch Normalization or other advanced sub multi task layers as seen in the recent state-of-the-art methods in section \ref{Chapter3:RelatedWork_NNModel}.

Each block of decoder comprises of 4 layers, starting with up-sampling by Bi linear with linear interpolation method. followed by concatenation operation and 2 convolutions 2D layers. The number of filters for these convolutions layer is decided by the number of filter obtained from the encoder layer which can be given by \(D_n =  {E_m} / {2^n}\) where \(D_n\) is number of filters at decoder block \(n\) and \(E_m\) is the number of filter present in the layer which is concatenated from encoder. The kernel size of the 2 convectional layers are of size (3$\times$3). The model proposed by  \cite{Alhashim2018} has target input dimension of 460$\times$640$\times$3 for RGB image and  dimension of 230$\times$320$\times$1 for ground truth label and we acquire the same principles for all our experiments. Also in the results section \ref{Chapter6:Results} we had some experiments by retraining the decodered part with different dataset and configurations to understand the influence of different 

In this approach we also use the transfer learning by obtaining weights which was provided by Alhashim et. al. \cite{Alhashim2018}. 


\section{Experimental Setup}
\label{Chapter5:Experimental_Setup}
\subsection{Evaluation Configurations}

Our entire experimental experimental configuration are described in the fig \ref{fig:Experimental_Setup}. All the configuration where made based on two model \textbf{A1} and \textbf{A2} and two data pre-processing \textbf{Holes} and \textbf{No\_Holes} which gives us 4 configuration to analyze the approach. 


First, Our proposed model \textbf{A1} is a simple U-Net style architecture - also can be categorized as encoder-decoder style of architecture. The reason for this model is to see if a simple network can learn the structural dependencies with respect to its depth. Knowing that the depth images are very much are highly structural dependent we wanted to see how well a network can learn without any prior knowledge. This motivated us to use an existing model to evaluate against the proposed model.We found out the our model performs relatively poor. We used the pre-trained weights of encoder and retrained the decoder layers with new structure dataset. And the results of this transfer leaning makes it more efficient. Hence we have 2 models, a simple U-Net with de-convolutional layers we call it as approach 1  (\textbf{A1}) and another model proposed by Alhashim et. al. \cite{Alhashim2018} and applied transfer learning approach we call it as approach 2 (\textbf{A2}) and \textbf{A2} become the main contribution for good results compared with U-Net approach model. 

Secondly, we wanted to investigate if a model can learn the holes generated from the SLB sensor. In our literature study we where unable to find previous work which was focused to make the network learn the holes and regenerated. The most common and widely used method was to either in-paint (or fill) the holes using neighboring pixel \cite{silberman11indoor} method. We approached this problem by saving the holes and map all the invalid holes to zero value instead of finding neighboring pixel which was used in NYU dataset. We trained \textbf{A2} model in two different fashion. One by mapping the holes to a zero value we call it as \textbf{A2-Holes} and another we find the neighboring pixels and we call this approach as \textbf{A2-NoHoles}. Note that, \textbf{A2-Holes} method might need some pre processing as we will will discuss more in the results section \ref{Chapter6:Results}.



\begin{figure}[h]
    \centering
    \includegraphics[width = 15cm]{Figures/config_setup.png}
    \caption{Different experimental configuration were setup based on \textbf{A1} and \textbf{A2} approaches}
    \label{fig:Experimental_Setup}
\end{figure}{}



Therefore in summary, we have two model \textbf{A1} and \textbf{A2} for validating the proposed model and effects of structural Characteristic. Further more \textbf{A2} model where retrained with two different modified input features for investigating on regeneration of holes. This gives us two more configuration of \textbf{A2} modelnamely \textbf{A2\_Holes} and \textbf{A2\_NoHoles}. 

\section{Evaluation Metrics}
For our evaluation we use standard metrics used in prior work \cite{Alhashim2018, eigen2014depth}. The error metrics are defined as follows:

\begin{itemize}
    
    \item Root Mean Squared Error (RMSE):
    \begin{equation} \label{RMSE}
        \sqrt{\frac{1}{n} \sum_{p}^{n}{(y_{p} - \hat{y}_{p}})^2}
    \end{equation}
    
    \item Average ({log_{10}}) error: 
    \begin{equation} \label{avg_log}
        \frac{1}{n} \sum_{p}^{n} \left|log_{10}(y_{p}) - log_{10}(\hat{y}_{p}) \right|
    \end{equation}
   
    \item Threshold Accuracy (\(\deta_{i}\)): \({\delta = max (\frac{{y_{p}}}{\hat{y_{p}}}, \frac{\hat{y_{p}}}{{y_{p}}})}\)
\end{itemize}{}

\begin{equation} \label{rms}
    {rms =  \frac{1}{n} \sum_{p}^{n} \frac{\left|y_{p} - \hat{y_{p}}\right|}{y}}
\end{equation}



 \({\frac{1}{n} \sum_{p}^{n} \left|log_{10}(y_{p}) - log_{10}(\hat{y_{p}}) \right|}\)

\begin{equation} \label{accuracy}
    {\deta = max (\frac{{y_{p}}}{\hat{y_{p}}}, \frac{\hat{y_{p}}}{{y_{p}}})}
\end{equation}


\newpage






We would like to address that out of the 6 standard error matrics we exclude Average Relative Error (rel): 


\section{Schedule and Milestones}
1 page: Provide a GANTT diagram based on week
 
% Chapter Template


\chapter{Results and Discussion}

\label{Chapter6:Results}

%----------------------------------------------------------------------------------------
% Transfer learning
% effect of loss function
% effect of camera intrinsic 
% effect of Hyper parameter
% Effect of recreating holes.. 
 
 
%----------------------------------------------------------------------------------------
In this chapter we have evaluated your system and presented a comparative results  based on different experimental setup discussed in section \ref{Chapter5:Methodology}. We have categorized our experiments into 3 different sections as shown in the table below \ref{table:Results_main} to investigate in three difference areas. This table lists all the experiments carried out for this study on depth estimation. In the following sections we will be describing our results mentioned in the table and will be discussing about it in its respective sections.  We have 3 accuracy values based on different threshold namely (a1, a2, and a2) and two error metrics namely root mean square error (rms) and log\_10 error as mentioned in section. Experiments \textbf{E1} and \textbf{E2} where performed to understand the importance of structural dependency of depth maps and to test our proposed model against the state of the art system. In experiments \textbf{E3} and \textbf{E4} we validate the influence of different camera intrinsic properties over data from structure sensor, by this we can also see the influence of transfer learning even when trained on different dataset with different feature properties. Experiments \textbf{E5} and \textbf{E6} was carried out to study behaviour of probabilistic distribution over the invalid pixel which we call it as holes, this leads us to answer the question of efficient way for depth estimation.

 
% Please add the following required packages to your document preamble:
% \usepackage{multirow}
\begin{table}[h]
\begin{tabular}{p{0.05\linewidth}p{0.3\linewidth}p{0.1\linewidth}p{0.1\linewidth}p{0.08\linewidth}p{0.08\linewidth}p{0.07\linewidth}}
\hline
\textbf{\#} & \textbf{Model} & \multicolumn{3}{l}{\textbf{Accuracy}} & \multicolumn{2}{l}{\textbf{Error}} \\ \cline{3-7} 
                    &                        & a1       & a2       & a3      & RMSE         & log\_10      \\ \hline
\multicolumn{7}{l}{\texttt{Influence of Structural Characteristics}}                                            \\ \hline
\textbf{E1}                  &  \textbf{A1}  & 0.22         & 0.43          &  0.61       & 0.34            &   0.001           \\ \hline
\textbf{E2}                  & \textbf{A2}  &    0.60  & 0.859 & 0.93       &   0.19          &0.11              \\ \hline
\multicolumn{7}{l}{\texttt{Influence Of Transfer Learning}}                                                                   \\ \hline
\textbf{E3}                  & \textbf{A2\_Holes}(N+S)              & 0.39   & 0.58   & 0.65  & 0.27      & 1.70       \\ \hline
\textbf{E4}                  & \textbf{A2\_Holes}(S) & 0.33   & 0.55   & 0.63  & 0.34      & 1.78       \\ \hline
\multicolumn{7}{l}{\texttt{Holes Regeneration}}                                                       \\ \hline
\textbf{E5}                  & \textbf{A2\_NoHoles}            & 0.98   & 0.98   & 0.98  & 0.10       & 0.18        \\ \hline
\textbf{E6}                  & \textbf{A2\_Holes}              & 0.39   & 0.58   & 0.65  & 0.27      & 1.70       \\ \hline
\end{tabular}

\caption{This table list all the results of experimental configuration performed. These experiments are grouped in to three categories. In E3, (N+S) denotes that model trained on NYU\_v2 and SD, where as in E4, (S) denotes model trained on SD\V2}
\label{table:Results_main}
\end{table}

\newpage

\section{Influence of Structural Characteristics}
 \begin{figure}[h]
\settoheight{\tempdima}{\includegraphics[width=.32\linewidth]{example-image-a}}%
\centering\begin{tabular}{@{}c@{ }c@{ }c@{ }c@{}}
&\textbf{RGB} & \textbf{Truth} & \textbf{Predticted} \\
\rowname{E1 (a)}&
\includegraphics[width=.3\linewidth]{Figures/results/s1_a1/u0RAW_RGB.png}&
\includegraphics[width=.3\linewidth]{Figures/results/s1_a1/u0Truth.png}&
\includegraphics[width=.3\linewidth]{Figures/results/s1_a1/u0Predicted.png}\\[-1ex]
\rowname{E2 (b)}&
\includegraphics[width=.3\linewidth]{Figures/results/s1_a1/0RAW_RGB.png}&
\includegraphics[width=.3\linewidth]{Figures/results/s1_a1/0Truth.png}&
\includegraphics[width=.3\linewidth]{Figures/results/s1_a1/0Predicted.png}\\[-1ex]
\end{tabular}
\caption{\textbf{Influence of Structural Characteristics:} All the \textbf{E3} methods are in different  }%
\label{fig:results_E1_E2}
\end{figure}

\label{Chapter6:Influence_Structural_Char}
In this section we have investigated the Influence of Structural Characteristics of a pre-trained network backbone. \textbf{E1 - A1} model has no previously trained weights whereas \textbf{E2- A2} model has DenseNet backbone with pre trained weights only for the encoder part, which means, by evaluating \textbf{E1 - A1} against \textbf{E2 - A2} we can find out wheather the pre-trained backbone encoder model can have a influence over our results. One main significant different is in the decoder block, \textbf{A2} model is not trained on ImageNet as we discussed earlier in Section \ref{Chapter3:RelatedWork_NNModel} which implies the \textbf{A2} encoder has learnt Structural Characteristics feature from RBG images. Therefor we evaluate this influence of such pre-trained backbone. Therefore we train both the models with same SD\_v1 dataset with the train, validation and test set split of 80\%, 10\%  and 10\% respectively. Note that in \textbf{E2 - A2} case, we train only the decoder part. From the result in table \ref{table:Results_main} we see that the model performs gives a accuracy \textbf{a1} of only \textbf{0.22} which is 22 \% when compared with \textbf{A2} accuracy of \textbf{0.60}. Also the RMSE error is also high in \textbf{E1 - A1} compared to \textbf{E2 - A2}. Similarly the output from both the model in fig. \ref{fig:results_E3_E4} also shows  \textbf{E1 - A1} failed to perform. Important factor to notice is that, there were several layer parameters and hyper parameters where changed and tested before reaching the current \textbf{E1 - A2}. Due to the poor performance we did not document every step of model tuning of \textbf{E1 - A1} 

Observation: On a positive note for \textbf{E1 - A1} there are two things to be noted. First,  \textbf{E1 - A1} model can learn some structure characters as seen in the predicted image in fig. \ref{fig:results_E3_E4} \textbf{E1(a)} we notice that the chair could be identified to be closer than the wall behind. Second, we noticed that training and prediction is relatively  faster than \textbf{E2 - A2} model which means smaller network can give faster results. On the negative  side performance is very poor. 
Therefore, it is very clear that \textbf{E2 - A2} with DenseNet backbone outperforms \textbf{E1 - A1}. Since the accuracy of \textbf{E1 - A1} was extremely poor we believe that this is an effect of less data available for training, which gave us an motivation for SD\_v2  dataset to generate more data. Thus we believe for further experiments choosing \textbf{A2} model over \textbf{A1} would be beneficial. Which leads us to the second step of this study which is to answer, if we can train the network to reproduce the holes in the similar fashion to Structure Sensor. And we have a better working model \textbf{A2} but still not the best. Since as we see in Fig. \ref{fig:results_E1_E2} the sooth edges around the objects in the predicted frame. 
 


 

 \section{Effect of Camera Properties}
 \label{Chapter6:Transfer_Learning}
Second investigation was performed to find the influence of transfer learning. In this method we input depth image for the network was with holes, still keeping the primary objective in this second stage of experimental process of finding optimal model for the best results. In order to find the best working model, From the previous investigation we understood the impact of pre-trained backbone (encoder) in our case which is DenseNet, in this experiment we two different configuration changes only on decoder part by keeping the encoder unchanged. Initially in first experiment \textbf{E3 - A2\_Holes(N+S)}  we trained our decoder part with two datasets one from kinect sonsor (NYU\_v2) and Structure sensor (SD\_v2)and in the second experiment \textbf{E4 - A2\_Holes(S)} we only with Structure sensor (SD\_v2) dataset. By this we can study the effects and influence different camera properties in our results there by achieving optimal model for our task.

The results from the table \ref{table:Results_main} we notice that all the three accuracy of \textbf{E3 - A2\_Holes(N+S)} is higher than \textbf{E4 - A2\_Holes(S)} similarly both the error is also lower for \textbf{E3 - A2\_Holes(N+S)} which means \textbf{E3 - A2\_Holes(N+S)} perform better than  \textbf{E4 - A2\_Holes(S)}. But when we compare difference of these five metrics individually, we find that there are only small difference. For instance lets us consider RMSE, \textbf{E3 - A2\_Holes(N+S)} has error rate of \textbf{0.27} while \textbf{E4 - A2\_Holes(S)} error rate is \textbf{0.34} with difference \textbf{0.07}. In the same way for accuracy \textbf{a1}, the accuracy difference between both the configurations where only \textbf{0.06}, for \textbf{a2} is \textbf{0.03} and for \textbf{a3} is just \textbf{0.02}. 

But when we compare the resultant depth maps from these two experimental configurations, visually the difference are higher with respect to precision at the object boundaries which can be seen in the Fig.\ref{fig:results_E3_E4}  where \textbf{E3 (a) - E3(c)} represented the resultant output from \textbf{E3 - A2\_Holes(N+S)} and similarly  \textbf{E4 (a) - E4(c)} represented the output from \textbf{E4 - A2\_Holes(S)}. We notice two significant difference, first a blurred effect at object edges for the model trained only on SD\_v2 dataset, where as we get refined edges with \textbf{E4 - A2\_Holes(S)}. Secondly, we observe that our model \textbf{E3 - A2\_Holes(N+S)} has learnt the holes. What is more interesting is as we see in the Fig. \ref{fig:results_E3_E4} \textbf{E3 (b)} model can learn even the complex environment which is comprised of many pole looking structures (legs of chairs in a classroom environment).

Oservation: As we have seen that our \textbf{A2} model when trained with both dataset performs better than when trained on one dataset alone. On a positive note, we see that data from Kinect sensor has improved our results. On the other hand, when we consider the size of SD\_v2 compared with NVU\_v2, SD\_v2 is relatively small. So it is difficult to study and conclude the effect of camera properties. But what we can say is NYU\_v2 dataset has a significant impact in our result and our current model can be trained to higher precision even for regeneration of holes.



 
 \begin{figure} [h]
\settoheight{\tempdima}{\includegraphics[width=.32\linewidth]{example-image-a}}%
\centering\begin{tabular}{@{}c@{ }c@{ }c@{ }c@{}}
&\textbf{RGB} & \textbf{Truth} & \textbf{Predticted} \\
\rowname{E3 (a)}&
\includegraphics[width=.3\linewidth]{Figures/results/s2_Holes/0RAW_RGB.png}&
\includegraphics[width=.3\linewidth]{Figures/results/s2_Holes/0Truth.png}&
\includegraphics[width=.3\linewidth]{Figures/results/s2_Holes/0Predicted.png}\\[-1ex]
\rowname{E3 (b)}&
\includegraphics[width=.3\linewidth]{Figures/results/s2_Holes/1RAW_RGB.png}&
\includegraphics[width=.3\linewidth]{Figures/results/s2_Holes/1Truth.png}&
\includegraphics[width=.3\linewidth]{Figures/results/s2_Holes/1Predicted.png}\\[-1ex]
\rowname{E3 (c)}&
\includegraphics[width=.3\linewidth]{Figures/results/s2_Holes/2RAW_RGB.png}&
\includegraphics[width=.3\linewidth]{Figures/results/s2_Holes/2Truth.png}&
\includegraphics[width=.3\linewidth]{Figures/results/s2_Holes/2Predicted.png}\\[-1ex]
\rowname{E4 (a)}&
\includegraphics[width=.3\linewidth]{Figures/results/s3_noNyu/0RAW_RGB.png}&
\includegraphics[width=.3\linewidth]{Figures/results/s3_noNyu/0Truth.png}&
\includegraphics[width=.3\linewidth]{Figures/results/s3_noNyu/0Predicted.png}\\[-1ex]
\rowname{E4 (b)}&
\includegraphics[width=.3\linewidth]{Figures/results/s3_noNyu/1RAW_RGB.png}&
\includegraphics[width=.3\linewidth]{Figures/results/s3_noNyu/1Truth.png}&
\includegraphics[width=.3\linewidth]{Figures/results/s3_noNyu/1Predicted.png}\\[-1ex]
\rowname{E4 (c)}&
\includegraphics[width=.3\linewidth]{Figures/results/s3_noNyu/2RAW_RGB.png}&
\includegraphics[width=.3\linewidth]{Figures/results/s3_noNyu/2Truth.png}&
\includegraphics[width=.3\linewidth]{Figures/results/s3_noNyu/2Predicted.png}\\[-1ex]
\end{tabular}
\caption{\textbf{Investigation on hole regeneration method:} All the \textbf{E3} methods are in different  }%
\label{fig:results_E3_E4}
\end{figure}





\newpage
 \section{Hole Regeneration}
 \label{Chapter6:Hole_Regeneration}
Further more to improve the results for estimation of depth maps for 3D reconstruction, we study the importance/influence of effects of a probabilistic distribution (in other words neural network's output is a distribution of probable values ranging from 0.0 to 1.0 on a image level) over this regenerating holes. As described in section \ref{Chapter4:Dataset}, in order to have a holes recreated we mapped all the dead value or no pixel region as zero. Therefore we have two methods of pre-possessing carried out with two different input feature type, one with holes and another without holes. We tested the performance of \textbf{A2\_Holes} against the \textbf{A2\_NoHoles} models. In Fig. \ref{fig:results_S2} (a) - (c) we see that there a good reconstruction of the depth map were all the holes were interpolated to its neighboring pixel.  Fig. \ref{fig:results_S2} (d) - (f)  corresponds to the model with the holes as a input to the network. As we see in Fig \ref{fig:results_S2}, we compare \textbf{E5} and \textbf{E6}. We have used two different color maps for \textbf{E5} and \textbf{E6}. This is because in \textbf{E4} we wanted to highlight the difference between holes which were mapped to zero and the closest region in an image. 

From the table \ref{table:Results_main} we notice that the on an average the model \textbf{E3 (A2\_NoHoles)} performs better. When RMSE is taken into consideration we can clearly see that \textbf{E3 (A2\_NoHoles)} performs better with the RMSE of \textbf{0.10} which is \textbf{0.17} lower than the \textbf{E4 (A2\_Holes)} with RMSE of \textbf{0.27}. And when we notice our accuracy a1, a2 and a3 we see the similar results when compared. But when notice in the Fig. \ref{fig:results_S2} (d) - (f) we have visually good prediction. when investigated further why such hugh difference in the accuracy and error we found out the error lies in the object boundaries. As we see in the fig () there is a interpolation intermediate pixel from the zero value (holes which are mapped as zeros) till the actual depth. This effect is cause because of the probabilistic distribution method of neural network characteristic. We strongly this is one of there main factors contributing to the difference in the error and accuracy between \textbf{E5} and \textbf{E6}. Now that we understand why such effects can be seen for the generation of holes, it is very important us to answer if such approach is beneficial or not. Given such a problem of linear \notice{interpolating prediction} one possible solotion can be given by some post-filtering or post processing methods but the idea for the case of this study is to provide end to end approch as much as possible thereby exploiting the potential of the neural network. 

Therefore we see that given an input with holes the model can learn the holes from the given monocular input. But it results into \notice{interpolating prediction} which we believe makes the 3D reconstruction of a scene more difficult and demands further post processing techniques. This leads us to an conclusion that interpolating of pixel is a better approach than making the network learn to predict holes. In addition to this, one of the other motivation to try this approach was to eliminate the wall effect of depth maps  for the pixel farther than the distance limit, one approach for this solution could be taking a threshold just before the highest pixels. This solves the problem of having wall in out 3D reconstruction scene.

In summary, we believe that the comparing the two different input representation, \textbf{A2\_NoHoles} approach where the holes are interpolated with its neighboring pixels is beneficial.





\begin{figure} 
\settoheight{\tempdima}{\includegraphics[width=.32\linewidth]{example-image-a}}%
\centering\begin{tabular}{@{}c@{ }c@{ }c@{ }c@{}}
&\textbf{RGB} & \textbf{Truth} & \textbf{Predticted} \\
\rowname{E3 (a)}&
\includegraphics[width=.3\linewidth]{Figures/results/s2_NoHoles/0RAW_RGB.png}&
\includegraphics[width=.3\linewidth]{Figures/results/s2_NoHoles/0Truth.png}&
\includegraphics[width=.3\linewidth]{Figures/results/s2_NoHoles/0Predicted.png}\\[-1ex]
\rowname{E3 (b)}&
\includegraphics[width=.3\linewidth]{Figures/results/s2_NoHoles/1RAW_RGB.png}&
\includegraphics[width=.3\linewidth]{Figures/results/s2_NoHoles/1Truth.png}&
\includegraphics[width=.3\linewidth]{Figures/results/s2_NoHoles/1Predicted.png}\\[-1ex]
\rowname{E3 (c)}&
\includegraphics[width=.3\linewidth]{Figures/results/s2_NoHoles/2RAW_RGB.png}&
\includegraphics[width=.3\linewidth]{Figures/results/s2_NoHoles/2Truth.png}&
\includegraphics[width=.3\linewidth]{Figures/results/s2_NoHoles/2Predicted.png}\\[-1ex]
\rowname{E4 (d)}&
\includegraphics[width=.3\linewidth]{Figures/results/s2_Holes/0RAW_RGB.png}&
\includegraphics[width=.3\linewidth]{Figures/results/s2_Holes/0Truth.png}&
\includegraphics[width=.3\linewidth]{Figures/results/s2_Holes/0Predicted.png}\\[-1ex]
\rowname{E4 (e)}&
\includegraphics[width=.3\linewidth]{Figures/results/s2_Holes/1RAW_RGB.png}&
\includegraphics[width=.3\linewidth]{Figures/results/s2_Holes/1Truth.png}&
\includegraphics[width=.3\linewidth]{Figures/results/s2_Holes/1Predicted.png}\\[-1ex]
\rowname{E4 (f)}&
\includegraphics[width=.3\linewidth]{Figures/results/s2_Holes/2RAW_RGB.png}&
\includegraphics[width=.3\linewidth]{Figures/results/s2_Holes/2Truth.png}&
\includegraphics[width=.3\linewidth]{Figures/results/s2_Holes/2Predicted.png}\\[-1ex]
\end{tabular}
\caption{\textbf{Investigation on hole regeneration method:} All the \textbf{E3} methods are in different  }%
\label{fig:results_S2}
\end{figure}





%\section{Qualitative evaluation}

%Here we have evaluted our result with the state of the art in the table below 
% Chapter Template


\chapter{Conclusion}
\label{Chaptee7:Concluion}
The primary objective of this study is to achieve a best working model for a specific environment and task based system which in our case is depth estimation using IPad and Structure sensor for indoor scene. Through the process of various experimental study we have achieved a best working model for depth estimation using Neural Network which can work as good as Structure Sensor with an accuracy score of \textbf{0.98}. The entire process of this study can be categorized in to three section.

First, we proposed a novel neural network architecture which is lower in complexity with respect to learning parameters. As an initial step we evaluated our proposed method against existing model architecture. Due to the lack of data to be trained our proposed model did not work as good as state of the art approach. The result of this is discussed in Section \ref{Chapter6:Influence_Structural_Char}, which also gives an understanding if the network with the pre-trained structural features can help in depth estimation. 

Second, we evaluated how well a model trained in Kinect sensor data can perform on Structure Sensor dataset thereby answering, if there a need for a task based model and dataset to be created or in other words what is the influence of different environment specific models. We retrained the existing model which was trained with Kinect Sensor with Structure Sensor data and tested their performance with each other (one model trained only on SD another model trained both on NYU\_v2 and SD). It turns out there is a hugh impact in terms of accuracy from the results discussed in Section \ref{Chapter6:Transfer_Learning} but when further tested against model only trained on NYU\_v2 dataset we see a significance impact of different camera properties which is discussed in Section \ref{Chapter6:ComapreS-F-A}. Thus from this, we conclude that a model must be tuned according to the given specific environment and task intended for the depth estimation case.

Thirdly, we proposed a novel idea to mimic the Structure Sensor by making the network learn the holes in the similar to sensor. Also keeping the focus of implementation of this depth estimation model for the task of 3D reconstruction in indoor environment we evaluated against standard approach of interpolating of holes to the nearest neighbour method. Model can learn and give a good prediction with high accuracy but with an artifact. We found that there is a intermediate pixel generated from the pixel till the next neighboring pixel which has valid depth (non zero pixel). The results are discussed in details in Section \ref{Chapter6:Hole_Regeneration}. This leads to a conclusion that we can only use such model with proper post processing methods. Since it is always desired to have an end to end approach we believe the better approach would be have label input for depth maps with no holes. 

Thus experimenting on different testing conditions we conclude that, having a pre-trained backbone for even with different task environment has significant improvement in learning structural dependency in depth estimation tasks but not the depth feature. There is a significance influence of different camera properties hence environment specific dataset must be need for better performance. Finally for 3D reconstruction, approach with no holes has proves to be beneficial to provide a end to end solution. 
% Chapter Template


\chapter{Image Gallery}

\label{Chapter8:Image_Gallery} 

%----------------------------------------------------------------------------------------
%	THESIS CONTENT - APPENDICES
%----------------------------------------------------------------------------------------

\appendix % Cue to tell LaTeX that the following "chapters" are Appendices

% Include the appendices of the thesis as separate files from the Appendices folder
% Uncomment the lines as you write the Appendices

%\bibliography{reference}
%% Appendix A

\chapter{Content of CD} % Main appendix title

\label{AppendixA} % For referencing this appendix elsewhere, use \ref{AppendixA}
We have provided with all the programming code used for this experiments with the python environment .yml file. Due to storage space constrain neural network model weights has been delivered to our advisor.


%\include{Appendices/AppendixB}
%\include{Appendices/AppendixC}

%----------------------------------------------------------------------------------------
%	BIBLIOGRAPHY
%----------------------------------------------------------------------------------------
%\bibliographystyle{ieeetr}
\printbibliography[heading=bibintoc]

%----------------------------------------------------------------------------------------

\end{document}  
