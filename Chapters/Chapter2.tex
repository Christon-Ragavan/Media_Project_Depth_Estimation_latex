% Chapter Template

\chapter{Basic Principles}
\label{Chapter2:Background} 

In this chapter we  discuss the basic  principles needed for this work. The entire chapter is divided into two sections. In the first Section we go through some theory of neural network and basic building blocks  needed for designing a proper neural network. Most of the concepts related to ANNs are taken from the book by Friedman et al. \cite{friedman2001elements}. The Second part will be comprising of some basics of depth estimation and camera parameters concepts. Since the area of Artificial Neural Network is very vast we will be focusing on the concepts which was specifically used for this study. 

\section{Artificial Neural Networks}

ANNs are developed algorithm similar to brain style computation which consists of different levels of neurons and capable of learning from different data. In other words we can say its a statistical learning or science of learning from given data. There are various types of learning approaches for solving problems regarding optimization . However, deep learning methodologies have proven to be favorable for many computer vision tasks related classification, detection, prediction etc.\cite{friedman2001elements} 

To understand the basics we first consider the two major types of Neural Networks which are supervised learning and unsupervised learning. In general, we denote \(X\) as input and \(Y\) as output of a given function. For supervised learning, the labels \(y_{i}\) for each input example \(x_{i}\) where    \(\{(x_{1},y_{1}),(x_{2},y_{2}),...(x_{n},y_{n})\} \in X\) for a learning algorithm satisfying function \(f:X \rightarrow Y\) and in such a way that the output \( \hat{y_{i}} = P (y_{i}|x_{i})\) where the output is depended upon the input. All labeled problems can be categorized under supervised learning. For  unsupervised learning, clustering or grouping is basically defined as the detection of similarities and joint density \(Pr(X)\) for a given input examples \({(x_{1},x_{2},...,x_{n})} \in X\) . In supervised learning there is a definite score of success which is \(P(y|y)\) \cite{friedman2001elements}. 

We can understand more about supervised learning with the help of simple binary logistic regression model which is given by
\begin{equation} \label{equation:feedforward}
    \hat{y_{i}} = x_{i}W + b 
\end{equation}

In our study we used supervised learning with convolutions network. To have a basic understanding, we discuss few nuts and blots concepts of CNN. The basic building block for a simple CNN is the filter. For the naming convention we use \textit{kernel size} for expressing the size of each filter which will be used throughout this thesis to denote filter size and not to be confused with number of filters. Idea of using filter could be easily understood when related to an example approach for finding edges in an image using a filter designed for edge detection or applying some Gaussian blur. But these filter which are represented in a matrix or in tensor form can be changed, updated or transformed based on \(W\) and \(b\) which is described in equation \ref{equation:feedforward}. Applying these filters in fashion of a sliding window, the network is converting an entire image to a space of the filter which depends on kernel size and also the step of the window slide which is also called as stride. For feature learning methods and filter space adaptation, various attribute contributes to this learning process such as forward propagation and weight \(W\) updates during back propagation which in turn modeled based on various loss function using optimization methods. An optimizer is basically a minimizing error function. In simple terms it is a error function which helps us to improve our results. Finally in order to tweak the training and optimize the learning progress we have various parameters called hyper-parameters. Some of these parameters are learning rate, in the case for gradient descent approach, decay rate which defined the descent rate after one iteration over entire training set, various regularizes and also some regularization term to prevent over fitting of the neural network model \cite{friedman2001elements}. 

%where b \in\mathbb{R}



\section{Depth Estimation}
3D spatial awareness have become more and more important with emerging technology of Augmented Reality and Virtual Reality. These 3D scanners use different approaches to estimate the depth out of a scene. Some of the major approaches being used are Stereo Imaging \cite{stereoimaging}, Structured Light System (SLS), Structure from Motion (SfM) \cite{sfm}, Time of Flight (ToF) \cite{timeofflight} and Laser Triangulation \cite{3DLasertechnique}. These approaches either use single frame or multiple frame to generate the third dimension Depth. For example, SfM perceives the depth using motion cues alone where it takes multiple frames and with different point of views of the object to determine the the motion cues\cite{sfm}. This not only needs heavy computation but is time consuming as well. On the other hand, Stereo Imaging computes a Disparity map by matching features of left and right image to calculate the depth of objects with the help of only two images \cite{stereoimaging}.

\begin{table}[h]
\begin{tabular}{@{}lll@{}}
\toprule
\textbf{Technique}                    & \textbf{Advantage}           & \textbf{Disadvantage}         \\ \midrule
Structure from Motion          & High capture frequency & Time consuming                 \\
Time of Flight       & No effect of lighting & Low capture frequency     \\
Structured Light System        & High resolution   & Prone to noise                 \\ 
                            &                     &                                               
\end{tabular}
\caption{Comparison of different 3D Scanning Techniques} 
\label{table:3DScanning}
\end{table}


ToF cameras work dynamically through the scene. They scan the environment using illuminating it with incoherent light. To measure the depth, the time taken by the light to reflect back to the sensor is noted. Although ToF cameras are fast, they are often not accurate. For more accuracy we need more time and hence it only works better with static objects. \cite{tof2}.The comparison between these techniques can also be seen in Table \ref{table:3DScanning} .These techniques are now widely being used to produce 3D scanners. For example Microsoft Kinect V2.0 uses ToF technique \cite{kinecttof}, while Structure Sensor by Occipital uses SLS \cite{Kalantari}.\\

Structure Sensor (2014) is an open source integration to mobile devices which can capture depth information using Structured Light System (SLS). It consist of a laser-emitting diode, infrared radiation range projector and an infrared sensor to sense the projected radiation. The infrared sensor records the reflecting intensity of the infrared (IR) light pattern projected by the IR projector onto the target while its system on the chip (SOC) triangulates the 3D scene  \cite{Kalantari} (As seen in Figure \ref{fig:Structuresensor}). Prime Sense chip (Heindl 2014) is used as the system on the chip (SOC) in Structure Sensor. The hardware can be easily installed to the mobile device through a customized USB 2.0 connection and Structure Software Development Kit. The micro lenses in the projector have different focal lengths which produces a non-uniform pattern of dots which varies at different distances \cite{Kalantari}. \\


\begin{figure}[h]
\centering
    \includegraphics[scale=0.29]{Figures/illustration-of-structure-sensor.png}
    \caption{Illustration of Structure Sensor}
    \label{fig:Structuresensor}
\end{figure}

The biggest advantage of Structure Sensor over any other available depth camera is the size of the hardware. As it is only 120mm long and 28mm wide, One can easily fit Structure Sensor in their pockets. Another advantage of it is that for small rooms and office environments it gives us very accurate results. Looking at our scope of research, we would also like to fully utilize the integration to a mobile device as well since our end goal is to make the network compatible to mobile devices. These features altogether makes Structure Sensor a perfect choice for our project.\\