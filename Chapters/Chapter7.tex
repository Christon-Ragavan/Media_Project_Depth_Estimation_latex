% Chapter Template


\chapter{Conclusion}
\label{Chaptee7:Concluion}
The primary objective of this study is to achieve a best working model for a specific environment and task based system which in our case is depth estimation using IPad and Structure sensor for indoor scene. Through the process of various experimental study we have achieved a best working model for depth estimation using Neural Network which can work as good as Structure Sensor with an accuracy score of \textbf{0.98}. Along with the accuracy score, we also studied its performance by visual representation in 2D and 3D. The entire process of this study can be categorized in to three section.

First, we proposed a novel neural network architecture which is lower in complexity with respect to learning parameters. As an initial step we evaluated our proposed method against existing model architecture. Due to the lack of data for training, our proposed model did not work as good as state of the art approach. The result of this is discussed in Section \ref{Chapter6:Influence_Structural_Char}, which also gives an understanding, if the network with the pre-trained structural features can help in depth estimation. 

Second, we evaluated how well a model trained in Kinect sensor data can perform on Structure Sensor dataset thereby answering, if there a need for a task based model and dataset to be created or in other words, the influence of different environment specific models. We retrained the existing model which was pre trained with Kinect Sensor with Structure Sensor data and tested their performance with each other (one model trained only on SD another model trained both on NYU\_v2 and SD). The resuls shows that, there is a huge impact in terms of accuracy from the results discussed in Section \ref{Chapter6:Transfer_Learning} but when further tested against model only trained on NYU\_v2 dataset we see a significance impact of different camera properties which is discussed in Section \ref{Chapter6:ComapreS-F-A}. Thus from this, we conclude that a model must be tuned according to the given specific environment and task intended for the depth estimation case.

Thirdly, we proposed a novel idea to mimic the Structure Sensor by making the network learn the holes in the similar to sensor. Also keeping the focus of implementation of this depth estimation model for the task of 3D reconstruction in indoor environment we evaluated against standard approach of interpolating of holes to the nearest neighbour method. Model can learn and give a good prediction with high accuracy but with an artifact. We found that there is a intermediate pixel generated. The results are discussed in details in Section \ref{Chapter6:Hole_Regeneration}. This leads to a conclusion that we can only use such model with proper post processing or pre processing methods. Since for our study, it is always desired to have an end to end approach we believe the best solution approach would be to have label input for depth maps with no holes. 

Thus experimenting on different testing conditions we conclude that, having a pre-trained backbone even when designed for different task and environment has significant improvement in learning structural dependency in depth estimation tasks but not the depth feature. This is because, there is a significance influence of different camera properties hence environment specific dataset must be needed for better performance. Finally for 3D reconstruction, approach with no holes has proves to be beneficial to provide a end to end solution.

